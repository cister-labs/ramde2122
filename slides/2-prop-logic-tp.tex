\documentclass[aspectratio=169]{beamer}

\input{macros/beamerconf}
\usepackage{etex} % fixes new-dimension error
\usepackage{lmodern} % fixes warnings
\usepackage{textcomp}% fixes warnings

\usepackage{graphicx,amsmath}
\usepackage{stmaryrd} % cf. interleave
%\usepackage{./macros/myisolatin1}
%\usepackage{./macros/prooftree}
\usepackage{alltt}
%\usepackage{./macros/circle}
\usepackage{listings}
\usepackage{relsize} % relative size fonts
\usepackage[normalem]{ulem} % strikethrough text (with \sout{.})
\usepackage{tikz}
\usetikzlibrary{%
  positioning
 ,patterns
 ,arrows
 ,automata
 ,calc
 ,shapes
 ,fit
 ,fadings
 ,decorations.pathreplacing
 ,plotmarks
% ,pgfplots.groupplots
 ,decorations.markings
}
% \tikzset{shorten >=1pt,node distance=2cm,on grid,auto,initial text={},inner sep=2pt}
\tikzstyle{aut}=[shorten >=1pt,node distance=2cm,on grid,auto,initial text={},inner sep=2pt]
\tikzstyle{st}=[circle,draw=black,fill=black!10,inner sep=3pt]
\tikzstyle{sst}=[rectangle,draw=none,fill=none,inner sep=3pt]
\tikzstyle{final}=[accepting]
\usepackage[normalem]{ulem} % striking out text with \sout{...}
\usepackage{xspace}

\usepackage{transparent}

% Nicer TT fonts
\usepackage[scaled=.83]{beramono}
\usepackage[T1]{fontenc}




%------ using eurosym -------------------------------------------------------
\usepackage{eurosym}
\def\inh#1{\mbox{\small\euro}_{#1}}
\def\eith#1#2{\mathopen{[}#1 \ ,#2\mathclose{]}}

%------ using xy ------------------------------------------------------------
\usepackage[all]{xy}
%\def\larrow#1#2#3{\xymatrix{ #3 & #1 \ar[l] _-{#2} }}
\def\larrow#1#2#3{\xymatrix{ #3 & #1 \ar[l] _--{#2} }}
\def\rarrow#1#2#3{\xymatrix{ #1 \ar[r]^-{#2} & #3 }}
\def\arLaw#1#2#3#4#5{
\xymatrix{
        #1      \ar@/^1pc/[rr]^-{#4} &
        #5 &
        #2      \ar@/^1pc/[ll]^-{#3}
}}
\def\arLeq#1#2#3#4{\arLaw{#1}{#2}{#3}{#4}\leq}
%------ using pstricks (rnode etc) ------------------------------------------
%\usepackage{pstricks,pst-node,pst-text,pst-3d}
\input{macros/macros}

\usepackage{macros/fitch}

\newcommand{\fitchr}[2]{\ensuremath{#1\text{\sc #2}}}

\usepackage{macros/bussproofs}


\setLecture{1}{RAMDE -- Requirements and Model-driven Engineering}

% \title{
% 	RAMDE -- Requirements and Model-driven Engineering 
% 	}
% \author{David Pereira \and Jos\'{e} Proen\c{c}a}
% \institute{CISTER -- ISEP \\ Porto, Portugal}
% \date{MScCCSE 2020/21}


\begin{document}

\frame[plain]{\titlepage}


\section{Propositional Logic }

\begin{slide}{Natural Deduction Rules}
\small

\begin{block}{Last RAMDE's class...}
  On the last class, you were introduced to Propositional Logic:
  \begin{itemize}
    \item its syntax and semantics
    \item normal forms: negative, disjunctive, and conjuntive
    \item rules for natural deduction
  \end{itemize}
\end{block}

\begin{block}{During this class...}
You will be exposed to the practice of construction proofs about Propositional Logic's formulae using Natural Deduction  
\end{block}


\begin{alert}{Warning:}
Becoming comfortable with this type of mathematics is not an easy task! Bare with me and be pattient! Train a lot by doing the exercises at home once again, to start solidifying the types of proof patterns that naturally will appear...
\end{alert}

\end{slide}


\begin{slide}{Recalling the rules of introduction and elimination: conjunction}

\begin{block}{Introduction}
\textit{If we know that both $\varphi$ and $\psi$ is hold, then so does their conjunction.}
  \begin{center}
    \begin{prooftree}
		\AxiomC{$\varphi$}
		\AxiomC{$\psi$}
		\RightLabel{\fitchr{\land}{I}}
		\BinaryInfC{$\varphi \land \psi$}
	\end{prooftree}
	\end{center}
\end{block}

\begin{block}{Elimination}
\textit{If know that $\varphi \land \psi$, then we can conclude that either of them also holds in isolation.}
  \begin{columns}
  \begin{column}{0.48\textwidth}
    \begin{prooftree}
		\AxiomC{$\varphi \land \psi$}
		\RightLabel{\fitchr{\land}{El}}
		\UnaryInfC{$\varphi$}
	\end{prooftree}
  \end{column}
  \begin{column}{0.48\textwidth}
    \begin{prooftree}
		\AxiomC{$\varphi \land \psi$}
		\RightLabel{\fitchr{\land}{Er}}
		\UnaryInfC{$\psi$}
	\end{prooftree}
  \end{column}
  \end{columns}  
\end{block}

\end{slide}

\begin{slide}{Quick exercise}

  \doExercise{Commutativity of $\land$}{Prove that if $\varphi \land \psi$ holds, then $\psi \land \varphi$ also holds. That is \color{orange}{$\varphi \land \psi \vdash \psi \land \varphi$}}
  \begin{center}
    \begin{fitch}
      \fj \varphi \land \psi \\
      \fa \varphi & \fitchr{\land}{El}(1) \\
      \fa \psi    & \fitchr{\land}{Er}(1) \\
      \fa \psi \land \varphi & \fitchr{\land}{I}(2,3)
    \end{fitch}
  \end{center}

\end{slide}



\begin{frame}
  \frametitle{Recalling the rules of introduction and elimination: disjunction}
  \begin{block}{Introduction}  
    {\it We can construct a new disjunction $\varphi \lor \psi$ if we know that either $\varphi$ or $\psi$ hold.}
    \begin{columns}
      \begin{column}{0.48\textwidth}
        \begin{prooftree}
	 	 \AxiomC{$\varphi$}
	 	 \RightLabel{\fitchr{\lor}{Il}}
		  \UnaryInfC{$\varphi \lor \psi$}
    	\end{prooftree}
      \end{column}
      \begin{column}{0.48\textwidth}
        \begin{prooftree}
		  \AxiomC{$\psi$}
		  \RightLabel{\fitchr{\lor}{Ir}}
		  \UnaryInfC{$\varphi \lor \psi$}
	   \end{prooftree}
      \end{column}
    \end{columns}
  \end{block}

  \begin{block}{Elimination}
	{\it The elimination, in this case, assumes the form of introducing a new formula $\theta$ in case we can derive $\theta$ from both $\varphi$ and $\psi$, and we know that $\varphi \lor \psi$ holds.}
    \begin{prooftree}
      \AxiomC{$\varphi \lor \psi$}
      \AxiomC{$[\varphi]$}
      \noLine
      \UnaryInfC{$\vdots$}
      \noLine
      \UnaryInfC{$\theta$}
      \AxiomC{$[\psi]$}
      \noLine
      \UnaryInfC{$\vdots$}
      \noLine
      \UnaryInfC{$\theta$}
      \RightLabel{{\fitchr{\lor}{E}}}
      \TrinaryInfC{$\theta$}
    \end{prooftree}
  \end{block}
\end{frame}

\begin{slide}{Quick exercise}

  \doExercise{Distributivity of disjunction}{
    Prove that if $(\varphi \lor \psi) \land \theta$ holds, then $(\varphi \land \theta) \lor (\psi \land \theta)$ also holds.}
  \begin{columns}
    \begin{column}{0.40\textwidth}
      \begin{fitch}
      \fj (\varphi \lor \psi) \land \theta \\
      \fa \varphi \lor \psi & \fitchr{\land}{El}(1) \\
      \fa \theta    & \fitchr{\land}{Er}(1) \\
      \fa \fj \varphi \\
      \fa \fa \varphi \land \theta & \fitchr{\land}{I}(4,3) \\
      \fa \fa (\varphi \land \theta) \lor (\psi \land \theta) & \fitchr{\lor}{Il}(5)
      \end{fitch}
    \end{column}
    \begin{column}{0.50\textwidth}
      \begin{fitch}
      \ftag{~}{\fa \vdots} \setcounter{fitchcounter}{6} \\
      \fa \fj \psi \\
      \fa \fa \psi \land \theta & \fitchr{\land}{I}(8,3) \\
      \fa \fa (\varphi \land \theta) \lor (\psi \land \theta) & \fitchr{\lor}{Ir}(8)\\
      \fa (\varphi \land \theta) \lor (\psi \land \theta) & \fitchr{\lor}{E}(2,4-6,7-9)
      
      \end{fitch}

    \end{column}  
  \end{columns}

\end{slide}


\begin{frame}
  \frametitle{Recalling the rules of introduction and elimination: Negation}
  \begin{block}{Introduction}
    \textit{If we can derive false from $\varphi$, then we can conclude that $\varphi$ does not hold, that is, its negation $\neg\varphi$ holds.}
    \begin{prooftree}
      \AxiomC{$[\varphi]$}
      \noLine
      \UnaryInfC{$\vdots$}
      \noLine
      \UnaryInfC{\textsc{F}}
      \RightLabel{\fitchr{\neg}{I}}
      \UnaryInfC{$\neg\varphi$}
    \end{prooftree}
  \end{block}
  \begin{block}{Elimination}
  \textit{If we know that $\neg\varphi$ is false, then ew can conclude that $\varphi$ holds.}
    \begin{prooftree}
      \AxiomC{$\neg\neg\varphi$}
      \RightLabel{\fitchr{\neg}{E}}
      \UnaryInfC{$\varphi$}
    \end{prooftree}
  \end{block}
\end{frame}

\begin{frame}
  \frametitle{Recalling the rules of introduction and elimination: False}
  \begin{block}{Introduction}
    \textit{If we assume $\varphi$ and, still, we are able to derive $\neg\varphi$, then we can conclude false. In fact, we found a contradiction!}
    \begin{prooftree}
      \AxiomC{$\varphi$}
      \noLine
      \UnaryInfC{$\vdots$}
      \noLine
      \UnaryInfC{$\neg\varphi$}
      \RightLabel{\fitchr{F}{I}}
      \UnaryInfC{{\sc F}}
    \end{prooftree}
  \end{block}
  \begin{block}{Elimination}
  \textit{From false, ew can conclude anything!}
    \begin{prooftree}
      \AxiomC{{\sc F}}
      \RightLabel{\fitchr{F}{E}}
      \UnaryInfC{$\varphi$}
    \end{prooftree}
  \end{block}
\end{frame}

\begin{slide}{Quick exercise}

  \doExercise{Negation and disjunction}{
    Prove that if $\neg(\varphi \lor \psi)$ holds, then $\neg\varphi \land \neg\psi$ also holds.}
  \begin{columns}
    \begin{column}{0.40\textwidth}
      \begin{fitch}
      \fj \neg(\varphi \lor \psi) \\
      \ftag{~}{\fa } \setcounter{fitchcounter}{1} \\
      \fa \fj \varphi & \fitchr{\land}{El}(1) \\
      \fa \fa \varphi \lor \psi & \fitchr{\lor}{Il}(2) \\
      \fa \fa \mathsc{F} & \fitchr{F}{I}(1,3) \\
      \fa \neg\varphi & \fitchr{\neg}{I}(2-4)
      \end{fitch}
    \end{column}
    \begin{column}{0.50\textwidth}
      \begin{fitch}
      \ftag{5}{\fj \psi \setcounter{fitchcounter}{5}}\\
      \fa \fa \varphi \lor \psi & \fitchr{\lor}{Ir}(6) \\
      \fa \fa \mathsc{F} & \fitchr{F}{I}(1,7) \\
      \fa \neg\psi & \fitchr{\neg}{I}(6-8)     \\
      \fa \color{green}{\neg\varphi} \land \neg\psi & \fitchr{\land}{I}(5,8)
      \end{fitch}
    \end{column}  
  \end{columns}

\end{slide}




\begin{frame}
  \frametitle{Natural Deduction Rules - Implication}
  \begin{block}{Introduction of implication}
    If we assume $\varphi$ and we can derive $\psi$ from it, then we can conclude that $\varphi \to \psi$.
    \begin{prooftree}
      \AxiomC{$[\varphi]$}
      \noLine
      \UnaryInfC{$\vdots$}
      \noLine
      \UnaryInfC{$\psi$}
      \RightLabel{$\to$-intro}
      \UnaryInfC{$\varphi \to \psi$}
    \end{prooftree}
  \end{block}
  \begin{block}{Elimination of implementation}
  From false, we can conclude whatever we want.
    \begin{prooftree}
      \AxiomC{$\varphi \to \psi$}
      \AxiomC{$\varphi$}
      \RightLabel{$\to$-elim}
      \BinaryInfC{$\psi$}
    \end{prooftree}
  \end{block}
\end{frame}



\begin{slide}{Quick exercise}

  \doExercise{Negation and disjunction}{
    Prove that if $(\varphi \lor \psi) \to \theta$ and $\varphi$ hold, then $\psi \to \theta$ also holds.}
\begin{center}
      \begin{fitch}
      \fa (\varphi \lor \psi) \to \theta \\
      \fj \varphi \\
      \ftag{~}{\fa } \setcounter{fitchcounter}{2} \\
      \fa \fj \psi \\
      \fa \fa \varphi \lor \psi & \fitchr{\lor}{Il}(2) \\
      \fa \fa \theta & \fitchr{\to}{E}(1,4) \\
      \fa \color{green}{\psi \to \theta} & \fitchr{\to}{I}(3-5)
      \end{fitch}
\end{center}
\end{slide}



\begin{slide}{How to create slides}
\small

\begin{itemize}
	\item \alert{\texttt{$\backslash$begin\{slide\}\{my title\} my content $\backslash$end\{slide\}}}
	\item \alert{\texttt{$\backslash$begin\{frame\} $\backslash$frametitle\{my title\} my content $\backslash$end\{frame\}}}
	\item \alert{\texttt{$\backslash$frame\{ $\backslash$frametitle\{my title\} my content \}}}
\end{itemize}
\end{slide}


\begin{slide}{Some handy macros and tips}
  \begin{itemize}
    \item Use beamer's macros \alert{\texttt{$\backslash$alert}} and \structure{\texttt{$\backslash$structure}} to color/highlight words.

    \item{Set the title of the lecture as in this file (using \alert{\texttt{$\backslash$setLecture}} with the lecture number) -- the number is used in the macros for exercises.}

		\item Make 1st title with \alert{\texttt{$\backslash$frame[plain]\{$\backslash$titlepage\}}}

		\item Use sections to separate topics.

    \item Use \alert{\texttt{$\backslash$frsplit\{left part\}\{right part\}}} to quickly start a 2 columns slide.
  \end{itemize}
\end{slide}

\begin{slide}{Macros for blocks and exercises}
	\begin{itemize}
    \item Use environment \alert{\texttt{$\backslash$begin\{block\}\{some title\}}} to have a gray block with a title.

    \item Use environment \alert{\texttt{$\backslash$begin\{exampleblock\}\{some title\}}} to present an example.

    \item Use environment \alert{\texttt{$\backslash$begin\{alertblock\}\{some title\}}} to present an example.

    \item Use macro \alert{\texttt{$\backslash$doExercise\{My title\}\{Exercise description\}}}.
    E.g.
    \doExercise{My title}{Exercise description}

    \item Use macro \alert{\texttt{$\backslash$exercise}} to start a numbered exercise inline.
    E.g.: \exercise Draw a diagram. \exercise Draw another one.

  \end{itemize}
  
\end{slide}


\end{document}
