\documentclass[aspectratio=169]{beamer}

\input{macros/beamerconf}
\usepackage{etex} % fixes new-dimension error
\usepackage{lmodern} % fixes warnings
\usepackage{textcomp}% fixes warnings

\usepackage{graphicx,amsmath}
\usepackage{stmaryrd} % cf. interleave
%\usepackage{./macros/myisolatin1}
%\usepackage{./macros/prooftree}
\usepackage{alltt}
%\usepackage{./macros/circle}
\usepackage{listings}
\usepackage{relsize} % relative size fonts
\usepackage[normalem]{ulem} % strikethrough text (with \sout{.})
\usepackage{tikz}
\usetikzlibrary{%
  positioning
 ,patterns
 ,arrows
 ,automata
 ,calc
 ,shapes
 ,fit
 ,fadings
 ,decorations.pathreplacing
 ,plotmarks
% ,pgfplots.groupplots
 ,decorations.markings
}
% \tikzset{shorten >=1pt,node distance=2cm,on grid,auto,initial text={},inner sep=2pt}
\tikzstyle{aut}=[shorten >=1pt,node distance=2cm,on grid,auto,initial text={},inner sep=2pt]
\tikzstyle{st}=[circle,draw=black,fill=black!10,inner sep=3pt]
\tikzstyle{sst}=[rectangle,draw=none,fill=none,inner sep=3pt]
\tikzstyle{final}=[accepting]
\usepackage[normalem]{ulem} % striking out text with \sout{...}
\usepackage{xspace}

\usepackage{transparent}

% Nicer TT fonts
\usepackage[scaled=.83]{beramono}
\usepackage[T1]{fontenc}




%------ using eurosym -------------------------------------------------------
\usepackage{eurosym}
\def\inh#1{\mbox{\small\euro}_{#1}}
\def\eith#1#2{\mathopen{[}#1 \ ,#2\mathclose{]}}

%------ using xy ------------------------------------------------------------
\usepackage[all]{xy}
%\def\larrow#1#2#3{\xymatrix{ #3 & #1 \ar[l] _-{#2} }}
\def\larrow#1#2#3{\xymatrix{ #3 & #1 \ar[l] _--{#2} }}
\def\rarrow#1#2#3{\xymatrix{ #1 \ar[r]^-{#2} & #3 }}
\def\arLaw#1#2#3#4#5{
\xymatrix{
        #1      \ar@/^1pc/[rr]^-{#4} &
        #5 &
        #2      \ar@/^1pc/[ll]^-{#3}
}}
\def\arLeq#1#2#3#4{\arLaw{#1}{#2}{#3}{#4}\leq}
%------ using pstricks (rnode etc) ------------------------------------------
%\usepackage{pstricks,pst-node,pst-text,pst-3d}
\input{macros/macros}

\usepackage{macros/fitch}

\newcommand{\fitchr}[2]{\ensuremath{#1\text{\sc #2}}}

\newcommand{\conji}[2]{\ensuremath{\land\text{\textbf{\textsc{I}}}(#1,#2)}}
\newcommand{\conjel}[1]{\ensuremath{\land\text{\textbf{\textsc{E}}}_l(#1)}}
\newcommand{\conjer}[1]{\ensuremath{\land\text{\textbf{\textsc{E}}}_r(#1)}}

\newcommand{\disjil}[1]{\ensuremath{\lor\text{\textbf{\textsc{I}}}_l(#1)}}
\newcommand{\disjir}[1]{\ensuremath{\lor\text{\textbf{\textsc{I}}}_r(#1)}}
\newcommand{\disje}[3]{\ensuremath{\lor\text{\textbf{\textsc{E}}}(#1,#2,#3)}}

\newcommand{\negi}[1]{\ensuremath{\neg\text{\textbf{\textsc{I}}}(#1)}}
\newcommand{\nege}[1]{\ensuremath{\neg\text{\textbf{\textsc{E}}}(#1)}}

\newcommand{\falsei}[2]{\ensuremath{\bot\text{\textbf{\textsc{I}}}(#1,#2)}}
\newcommand{\falsee}[1]{\ensuremath{\bot\text{\textbf{\textsc{E}}}(#1)}}

\newcommand{\impi}[2]{\ensuremath{{\to}\text{\textbf{\textsc{I}}}(#1{-}#2)}}
\newcommand{\impe}[2]{\ensuremath{{\to}\text{\textbf{\textsc{E}}}(#1,#2)}}


\usepackage{macros/bussproofs}
\usepackage{bm}


\setLecture{1}{RAMDE -- Requirements and Model-driven Engineering}

% \title{
% 	RAMDE -- Requirements and Model-driven Engineering 
% 	}
% \author{David Pereira \and Jos\'{e} Proen\c{c}a}
% \institute{CISTER -- ISEP \\ Porto, Portugal}
% \date{MScCCSE 2020/21}


\begin{document}

\frame[plain]{\titlepage}


\section{Propositional Logic }

\begin{slide}{Natural Deduction Rules}
\small

\begin{block}{Last RAMDE's class...}
  On the last class, you were introduced to Propositional Logic:
  \begin{itemize}
    \item its syntax and semantics
    \item normal forms: negative, disjunctive, and conjuntive
    \item rules for natural deduction
  \end{itemize}
\end{block}

\begin{block}{During this class...}
You will be exposed to the practice of construction proofs about Propositional Logic's formulae using Natural Deduction  
\end{block}


\begin{alert}{Warning:}
Becoming comfortable with this type of mathematics is not an easy task! Bare with me and be pattient! Train a lot by doing the exercises at home once again, to start solidifying the types of proof patterns that naturally will appear...
\end{alert}

\end{slide}


\begin{slide}{Recalling the rules of introduction and elimination: conjunction}

\begin{block}{Introduction}
\textit{If we know that both $\varphi$ and $\psi$ is hold, then so does their conjunction.}
  \begin{center}
    \begin{prooftree}
		\AxiomC{$\varphi$}
		\AxiomC{$\psi$}
		\RightLabel{$\land\text{\textbf{\textsc{I}}}$}
		\BinaryInfC{$\varphi \land \psi$}
	\end{prooftree}
	\end{center}
\end{block}

\begin{block}{Elimination}
\textit{If know that $\varphi \land \psi$, then we can conclude that either of them also holds in isolation.}
  \begin{columns}
  \begin{column}{0.48\textwidth}
    \begin{prooftree}
		\AxiomC{$\varphi \land \psi$}
		\RightLabel{$\land\text{\textbf{\textsc{E}}}_l$}
		\UnaryInfC{$\varphi$}
	\end{prooftree}
  \end{column}
  \begin{column}{0.48\textwidth}
    \begin{prooftree}
		\AxiomC{$\varphi \land \psi$}
		\RightLabel{$\land\text{\textbf{\textsc{E}}}_r$}
		\UnaryInfC{$\psi$}
	\end{prooftree}
  \end{column}
  \end{columns}  
\end{block}

\end{slide}

\begin{slide}{Quick exercise}

  \doExercise{Commutativity of $\land$}{Prove that if $\varphi \land \psi$ holds, then $\psi \land \varphi$ also holds. That is $\varphi \land \psi \vdash \psi \land \varphi$}
  \begin{center}
    \begin{fitch}
      \fj \varphi \land \psi \\
      \fa \varphi & \conjel{1} \\
      \fa \psi    & \conjer{1} \\
      \fa \color{orange}{\psi \land \varphi} & \conji{2}{3}
    \end{fitch}
  \end{center}

\end{slide}



\begin{frame}
  \frametitle{Recalling the rules of introduction and elimination: disjunction}
  \begin{block}{Introduction}  
    {\it We can construct a new disjunction $\varphi \lor \psi$ if we know that either $\varphi$ or $\psi$ hold.}
    \begin{columns}
      \begin{column}{0.48\textwidth}
        \begin{prooftree}
	 	 \AxiomC{$\varphi$}
	 	 \RightLabel{$\lor\text{\textbf{\textsc{I}}}_l$}
		  \UnaryInfC{$\varphi \lor \psi$}
    	\end{prooftree}
      \end{column}
      \begin{column}{0.48\textwidth}
        \begin{prooftree}
		  \AxiomC{$\psi$}
		  \RightLabel{$\lor\text{\textbf{\textsc{I}}}_r$}
		  \UnaryInfC{$\varphi \lor \psi$}
	   \end{prooftree}
      \end{column}
    \end{columns}
  \end{block}

  \begin{block}{Elimination}
	{\it The elimination, in this case, assumes the form of introducing a new formula $\theta$ in case we can derive $\theta$ from both $\varphi$ and $\psi$, and we know that $\varphi \lor \psi$ holds.}
    \begin{prooftree}
      \AxiomC{$\varphi \lor \psi$}
      \AxiomC{$[\varphi]$}
      \noLine
      \UnaryInfC{$\vdots$}
      \noLine
      \UnaryInfC{$\theta$}
      \AxiomC{$[\psi]$}
      \noLine
      \UnaryInfC{$\vdots$}
      \noLine
      \UnaryInfC{$\theta$}
      \RightLabel{$\lor\text{\textbf{\textsc{E}}}$}
      \TrinaryInfC{$\theta$}
    \end{prooftree}
  \end{block}
\end{frame}

\begin{slide}{Quick exercise}

  \doExercise{Distributivity of disjunction}{
    Prove that if $(\varphi \lor \psi) \land \theta$ holds, then $(\varphi \land \theta) \lor (\psi \land \theta)$ also holds.}
  \begin{columns}
    \begin{column}{0.40\textwidth}
      \begin{fitch}
      \fj (\varphi \lor \psi) \land \theta \\
      \fa \varphi \lor \psi & \conjel{1} \\
      \fa \theta    & \conjer{1} \\
      \fa \fj \varphi \\
      \fa \fa \varphi \land \theta & \conji{4}{3} \\
      \fa \fa (\varphi \land \theta) \lor (\psi \land \theta) & \disjil{5} \\
      \ftag{~}{\fa \vdots} \\
      \end{fitch}
    \end{column}
    \begin{column}{0.50\textwidth}
      \begin{fitch}
      \ftag{~}{\fa \vdots} \setcounter{fitchcounter}{6} \\
      \fa \fj \psi \\
      \fa \fa \psi \land \theta & \conji{8}{3} \\
      \fa \fa (\varphi \land \theta) \lor (\psi \land \theta) & \disjir{8}\\
      \fa \color{orange}{(\varphi \land \theta) \lor (\psi \land \theta)} & \disje{2}{4{-}6}{7{-}9}
      \end{fitch}
    \end{column}  
  \end{columns}

\end{slide}


\begin{frame}
  \frametitle{Recalling the rules of introduction and elimination: Negation}
  \begin{block}{Introduction}
    \textit{If we can derive false from $\varphi$, then we can conclude that $\varphi$ does not hold, that is, its negation $\neg\varphi$ holds.}
    \begin{prooftree}
      \AxiomC{$[\varphi]$}
      \noLine
      \UnaryInfC{$\vdots$}
      \noLine
      \UnaryInfC{$\bm{\bot}$}
      \RightLabel{$\neg\text{\textbf{\textsc{I}}}$}
      \UnaryInfC{$\neg\varphi$}
    \end{prooftree}
  \end{block}
  \begin{block}{Elimination}
  \textit{If we know that $\neg\varphi$ is false, then ew can conclude that $\varphi$ holds.}
    \begin{prooftree}
      \AxiomC{$\neg\neg\varphi$}
      \RightLabel{$\neg\text{\textbf{\textsc{E}}}$}
      \UnaryInfC{$\varphi$}
    \end{prooftree}
  \end{block}
\end{frame}

\begin{frame}
  \frametitle{Recalling the rules of introduction and elimination: False}
  \begin{block}{Introduction}
    \textit{If we assume $\varphi$ and, still, we are able to derive $\neg\varphi$, then we can conclude false. In fact, we found a contradiction!}
    \begin{prooftree}
      \AxiomC{$\varphi$}
      \noLine
      \UnaryInfC{$\vdots$}
      \noLine
      \UnaryInfC{$\neg\varphi$}
      \RightLabel{$\bot\text{\textbf{\textsc{I}}}$}
      \UnaryInfC{$\bm{\bot}$}
    \end{prooftree}
  \end{block}
  \begin{block}{Elimination}
  \textit{From false, ew can conclude anything!}
    \begin{prooftree}
      \AxiomC{$\bm{\bot}$}
      \RightLabel{$\bot\text{\textbf{\textsc{E}}}$}
      \UnaryInfC{$\varphi$}
    \end{prooftree}
  \end{block}
\end{frame}

\begin{slide}{Quick exercise}

  \doExercise{Negation and disjunction}{
    Prove that if $\neg(\varphi \lor \psi)$ holds, then $\neg\varphi \land \neg\psi$ also holds.}
  
  \begin{columns}
    \begin{column}{0.40\textwidth}
      \begin{fitch}
      \fj \neg(\varphi \lor \psi) \\
      \ftag{~}{\fa } \setcounter{fitchcounter}{1} \\
      \fa \fj \varphi & \conjel{1} \\
      \fa \fa \varphi \lor \psi & \disjil{2} \\
      \fa \fa \bm{\bot} & \falsei{1}{3} \\
      \fa \neg\varphi & \negi{2{-}4}
      \end{fitch}
    \end{column}
    \begin{column}{0.50\textwidth}
      \begin{fitch}
      \ftag{6}{\fj \psi \setcounter{fitchcounter}{6}}\\
      \fa \fa \varphi \lor \psi & \disjir{6} \\
      \fa \fa \bm{\bot} & \falsei{1}{7} \\
      \fa \neg\psi & \negi{6{-}8}     \\
      \fa \color{orange}{\neg\varphi} \land \neg\psi & \conji{5}{8}
      \end{fitch}
    \end{column}  
  \end{columns}

\end{slide}




\begin{frame}
  \frametitle{Natural Deduction Rules - Implication}
  \begin{block}{Introduction of implication}
    If we assume $\varphi$ and we can derive $\psi$ from it, then we can conclude that $\varphi \to \psi$.
    \begin{prooftree}
      \AxiomC{$[\varphi]$}
      \noLine
      \UnaryInfC{$\vdots$}
      \noLine
      \UnaryInfC{$\psi$}
      \RightLabel{${\to}\text{\textbf{\textsc{I}}}$}
      \UnaryInfC{$\varphi \to \psi$}
    \end{prooftree}
  \end{block}
  \begin{block}{Elimination of implementation}
  From false, we can conclude whatever we want.
    \begin{prooftree}
      \AxiomC{$\varphi \to \psi$}
      \AxiomC{$\varphi$}
      \RightLabel{${\to}\text{\textbf{\textsc{E}}}$}
      \BinaryInfC{$\psi$}
    \end{prooftree}
  \end{block}
\end{frame}



\begin{slide}{Quick exercise}

  \doExercise{Negation and disjunction}{
    Prove that if $(\varphi \lor \psi) \to \theta$ and $\varphi$ hold, then $\psi \to \theta$ also holds.}
\begin{center}
      \begin{fitch}
      \fa (\varphi \lor \psi) \to \theta \\
      \fj \varphi \\
      \ftag{~}{\fa } \setcounter{fitchcounter}{2} \\
      \fa \fj \psi \\
      \fa \fa \varphi \lor \psi & \disjil{2} \\
      \fa \fa \theta & \impe{1}{4} \\
      \fa \color{orange}{\psi \to \theta} & \impi{3}{5}
      \end{fitch}
\end{center}
\end{slide}

\begin{frame}
  \frametitle{Natural Deduction Rules - Derived rules}
  \begin{columns}
      \begin{column}{0.48\textwidth}
        \begin{prooftree}
	 	 \AxiomC{$\varphi \to \psi$}
	 	 \AxiomC{$\neg\psi$}
	 	 \RightLabel{\textbf{\textsc{MT}}}
		  \BinaryInfC{$\neg\varphi$}
    	\end{prooftree}
      \end{column}
      \begin{column}{0.48\textwidth}
        \begin{prooftree}
		  \AxiomC{$\varphi$}
		  \RightLabel{$\neg\neg$\textbf{\textsc{I}}}
		  \UnaryInfC{$\neg\neg\varphi$}
	   \end{prooftree}
      \end{column}
    \end{columns}
    \begin{columns}
      \begin{column}{0.48\textwidth}
      \begin{prooftree}
        \AxiomC{$[\neg\varphi]$}
        \noLine
        \UnaryInfC{$\vdots$}
        \noLine
        \UnaryInfC{$\bm{\bot}$}
        \RightLabel{\textbf{\textsc{RA}}}
        \UnaryInfC{$\varphi$}
      \end{prooftree}
      \end{column}
      \begin{column}{0.48\textwidth}
        \begin{prooftree}
		  \AxiomC{$$}
		  \RightLabel{\textbf{\textsc{ET}}}
		  \UnaryInfC{$\varphi \lor \neg\varphi$}
	   \end{prooftree}
      \end{column}
    \end{columns}
\end{frame}

\begin{slide}{Lets prove the derived rules?}
  \doExercise{Modus Tolens}{Prove that $\varphi \to \psi, \neg\psi \vdash \neg\varphi$}
  \begin{fitch} 
    \fa \varphi \to \psi \\
    \fj \neg\psi \\
    \ftag{~}{\fa } \setcounter{fitchcounter}{2} \\
    \fa \fj \varphi \\
    \fa \fa \psi & \impe{1}{3} \\
    \fa \fa \bm{\bot} & \falsei{3}{4} \\
    \fa \color{orange}{\neg\varphi} & \negi{3{-}5}
  \end{fitch}
\end{slide}

\begin{slide}{Lets prove the derived rules?}
  \doExercise{Double Negation Introduction}{Prove that $\varphi \vdash \neg\neg\varphi$}
  \begin{fitch} 
    \fj \varphi \\
    \ftag{~}{\fa } \setcounter{fitchcounter}{1} \\
    \fa \fj \neg\varphi \\
    \fa \fa F & \falsei{1}{2} \\
    \fa \color{orange}{\neg\neg\varphi} & \negi{2{-}3}
  \end{fitch}
\end{slide}

\begin{slide}{Lets prove the derived rules?}
  \doExercise{Reductio ad absurdum}{Prove that $\neg\varphi \to F \vdash \varphi$}
  \begin{fitch} 
    \fj \neg\varphi \to F \\
    \ftag{~}{\fa } \setcounter{fitchcounter}{1} \\
    \fa \fj \neg\varphi \\
    \fa \fa F & \falsei{1}{2} \\
    \fa \neg\neg\varphi & \negi{2{-}3} \\
    \fa \color{orange}{\varphi} & \nege{4}
  \end{fitch}
\end{slide}

\begin{slide}{Lets prove the derived rules?}
  \doExercise{Excluded third}{Whatever $\varphi$ we have $\vdash \varphi \lor \neg\varphi$}
  \begin{columns}
      \begin{column}{0.48\textwidth}
      \begin{fitch} 
        \fa \fj \neg(\varphi \lor \neg\varphi)\\
        \ftag{~}{\fa \fa} \setcounter{fitchcounter}{1} \\
        \fa \fa \fj \varphi \\
        \fa \fa \fa \varphi \lor \neg\varphi & \disjir{2}\\
        \fa \fa \fa \bm{\bot} & \falsei{1}{3} \\
        \fa \fa \neg\varphi & \negi{2{-}4} \\
        \fa \fa \varphi \lor \neg\varphi & \disjil{5}
      \end{fitch}
      \end{column}
      \begin{column}{0.48\textwidth}
      \begin{fitch}
        \ftag{~}{\fa \fa \vdots} \setcounter{fitchcounter}{6} \\
        \fa \fa \bm{\bot} & \falsei{1}{6} \\
        \fa \neg\neg(\varphi \lor \neg\varphi) & \negi{1{-}7} \\
        \fa \color{orange}{\varphi \lor \neg\varphi} & \nege{8}
      \end{fitch}
      \end{column}
  \end{columns}
\end{slide}


\begin{slide}{Lets continue with more exercises}

  \doExercise{A first list of exercises}{Build the derivations for each of the statements below:
  \begin{itemize}
    \item $\vdash (\varphi \land \psi) \to \psi$
    \item $\vdash \varphi \to (\varphi \lor \psi)$
    \item $\vdash (\varphi \lor \psi) \to (\psi \lor \varphi)$
    \item $\theta \to (\varphi \to \psi), \neg\psi, \theta \vdash \neg\varphi$
    \item $\theta, \neg\varphi \vdash\neg(\theta \to \varphi)$
    \item $(\psi \land \theta) \to \neg\delta, \varphi \to \delta, \theta, \varphi \vdash \neg\psi$
    \item $(\psi \to \varphi) \land (\varphi \to \psi) \vdash (\varphi \land \psi) \lor (\neg\varphi \land \neg\psi)$
  \end{itemize}
  }
  
\end{slide}



%\begin{slide}{How to create slides}
%\small
%
%\begin{itemize}
%	\item \alert{\texttt{$\backslash$begin\{slide\}\{my title\} my content $\backslash$end\{slide\}}}
%	\item \alert{\texttt{$\backslash$begin\{frame\} $\backslash$frametitle\{my title\} my content $\backslash$end\{frame\}}}
%	\item \alert{\texttt{$\backslash$frame\{ $\backslash$frametitle\{my title\} my content \}}}
%\end{itemize}
%\end{slide}
%
%
%\begin{slide}{Some handy macros and tips}
%  \begin{itemize}
%    \item Use beamer's macros \alert{\texttt{$\backslash$alert}} and \structure{\texttt{$\backslash$structure}} to color/highlight words.
%
%    \item{Set the title of the lecture as in this file (using \alert{\texttt{$\backslash$setLecture}} with the lecture number) -- the number is used in the macros for exercises.}
%
%		\item Make 1st title with \alert{\texttt{$\backslash$frame[plain]\{$\backslash$titlepage\}}}
%
%		\item Use sections to separate topics.
%
%    \item Use \alert{\texttt{$\backslash$frsplit\{left part\}\{right part\}}} to quickly start a 2 columns slide.
%  \end{itemize}
%\end{slide}
%
%\begin{slide}{Macros for blocks and exercises}
%	\begin{itemize}
%    \item Use environment \alert{\texttt{$\backslash$begin\{block\}\{some title\}}} to have a gray block with a title.
%
%    \item Use environment \alert{\texttt{$\backslash$begin\{exampleblock\}\{some title\}}} to present an example.
%
%    \item Use environment \alert{\texttt{$\backslash$begin\{alertblock\}\{some title\}}} to present an example.
%
%    \item Use macro \alert{\texttt{$\backslash$doExercise\{My title\}\{Exercise description\}}}.
%    E.g.
%    \doExercise{My title}{Exercise description}
%
%    \item Use macro \alert{\texttt{$\backslash$exercise}} to start a numbered exercise inline.
%    E.g.: \exercise Draw a diagram. \exercise Draw another one.
%
%  \end{itemize}
%  
%\end{slide}


\end{document}
