\documentclass{beamer}

\input{macros/beamerconf}
\usepackage{etex} % fixes new-dimension error
\usepackage{lmodern} % fixes warnings
\usepackage{textcomp}% fixes warnings

\usepackage{graphicx,amsmath}
\usepackage{stmaryrd} % cf. interleave
%\usepackage{./macros/myisolatin1}
%\usepackage{./macros/prooftree}
\usepackage{alltt}
%\usepackage{./macros/circle}
\usepackage{listings}
\usepackage{relsize} % relative size fonts
\usepackage[normalem]{ulem} % strikethrough text (with \sout{.})
\usepackage{tikz}
\usetikzlibrary{%
  positioning
 ,patterns
 ,arrows
 ,automata
 ,calc
 ,shapes
 ,fit
 ,fadings
 ,decorations.pathreplacing
 ,plotmarks
% ,pgfplots.groupplots
 ,decorations.markings
}
% \tikzset{shorten >=1pt,node distance=2cm,on grid,auto,initial text={},inner sep=2pt}
\tikzstyle{aut}=[shorten >=1pt,node distance=2cm,on grid,auto,initial text={},inner sep=2pt]
\tikzstyle{st}=[circle,draw=black,fill=black!10,inner sep=3pt]
\tikzstyle{sst}=[rectangle,draw=none,fill=none,inner sep=3pt]
\tikzstyle{final}=[accepting]
\usepackage[normalem]{ulem} % striking out text with \sout{...}
\usepackage{xspace}

\usepackage{transparent}

% Nicer TT fonts
\usepackage[scaled=.83]{beramono}
\usepackage[T1]{fontenc}




%------ using eurosym -------------------------------------------------------
\usepackage{eurosym}
\def\inh#1{\mbox{\small\euro}_{#1}}
\def\eith#1#2{\mathopen{[}#1 \ ,#2\mathclose{]}}

%------ using xy ------------------------------------------------------------
\usepackage[all]{xy}
%\def\larrow#1#2#3{\xymatrix{ #3 & #1 \ar[l] _-{#2} }}
\def\larrow#1#2#3{\xymatrix{ #3 & #1 \ar[l] _--{#2} }}
\def\rarrow#1#2#3{\xymatrix{ #1 \ar[r]^-{#2} & #3 }}
\def\arLaw#1#2#3#4#5{
\xymatrix{
        #1      \ar@/^1pc/[rr]^-{#4} &
        #5 &
        #2      \ar@/^1pc/[ll]^-{#3}
}}
\def\arLeq#1#2#3#4{\arLaw{#1}{#2}{#3}{#4}\leq}
%------ using pstricks (rnode etc) ------------------------------------------
%\usepackage{pstricks,pst-node,pst-text,pst-3d}
\input{macros/macros}


\setLecture{1}{RAMDE -- Requirements and Model-driven Engineering}

% \title{
% 	RAMDE -- Requirements and Model-driven Engineering 
% 	}
% \author{David Pereira \and Jos\'{e} Proen\c{c}a}
% \institute{CISTER -- ISEP \\ Porto, Portugal}
% \date{MScCCSE 2020/21}


\begin{document}

\frame[plain]{\titlepage}


\section{Tips on how to write slides}


\begin{slide}{How to create slides}
\small

\begin{itemize}
	\item \alert{\texttt{$\backslash$begin\{slide\}\{my title\} my content $\backslash$end\{slide\}}}
	\item \alert{\texttt{$\backslash$begin\{frame\} $\backslash$frametitle\{my title\} my content $\backslash$end\{frame\}}}
	\item \alert{\texttt{$\backslash$frame\{ $\backslash$frametitle\{my title\} my content \}}}
\end{itemize}
\end{slide}


\begin{slide}{Some handy macros and tips}
  \begin{itemize}
    \item Use beamer's macros \alert{\texttt{$\backslash$alert}} and \structure{\texttt{$\backslash$structure}} to color/highlight words.

    \item{Set the title of the lecture as in this file (using \alert{\texttt{$\backslash$setLecture}} with the lecture number) -- the number is used in the macros for exercises.}

		\item Make 1st title with \alert{\texttt{$\backslash$frame[plain]\{$\backslash$titlepage\}}}

		\item Use sections to separate topics.

    \item Use \alert{\texttt{$\backslash$frsplit\{left part\}\{right part\}}} to quickly start a 2 columns slide.
  \end{itemize}
\end{slide}

\begin{slide}{Macros for blocks and exercises}
	\begin{itemize}
    \item Use environment \alert{\texttt{$\backslash$begin\{block\}\{some title\}}} to have a gray block with a title.

    \item Use environment \alert{\texttt{$\backslash$begin\{exampleblock\}\{some title\}}} to present an example.

    \item Use environment \alert{\texttt{$\backslash$begin\{alertblock\}\{some title\}}} to present an example.

    \item Use macro \alert{\texttt{$\backslash$doExercise\{My title\}\{Exercise description\}}}.
    E.g.
    \doExercise{My title}{Exercise description}

    \item Use macro \alert{\texttt{$\backslash$exercise}} to start a numbered exercise inline.
    E.g.: \exercise Draw a diagram. \exercise Draw another one.

  \end{itemize}
  
\end{slide}


\end{document}
