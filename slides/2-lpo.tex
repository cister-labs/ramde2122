\documentclass[aspectratio=169]{beamer}

\input{macros/beamerconf}
\usepackage{etex} % fixes new-dimension error
\usepackage{lmodern} % fixes warnings
\usepackage{textcomp}% fixes warnings

\usepackage{graphicx,amsmath}
\usepackage{stmaryrd} % cf. interleave
%\usepackage{./macros/myisolatin1}
%\usepackage{./macros/prooftree}
\usepackage{alltt}
%\usepackage{./macros/circle}
\usepackage{listings}
\usepackage{relsize} % relative size fonts
\usepackage[normalem]{ulem} % strikethrough text (with \sout{.})
\usepackage{tikz}
\usetikzlibrary{%
  positioning
 ,patterns
 ,arrows
 ,automata
 ,calc
 ,shapes
 ,fit
 ,fadings
 ,decorations.pathreplacing
 ,plotmarks
% ,pgfplots.groupplots
 ,decorations.markings
}
% \tikzset{shorten >=1pt,node distance=2cm,on grid,auto,initial text={},inner sep=2pt}
\tikzstyle{aut}=[shorten >=1pt,node distance=2cm,on grid,auto,initial text={},inner sep=2pt]
\tikzstyle{st}=[circle,draw=black,fill=black!10,inner sep=3pt]
\tikzstyle{sst}=[rectangle,draw=none,fill=none,inner sep=3pt]
\tikzstyle{final}=[accepting]
\usepackage[normalem]{ulem} % striking out text with \sout{...}
\usepackage{xspace}

\usepackage{transparent}

% Nicer TT fonts
\usepackage[scaled=.83]{beramono}
\usepackage[T1]{fontenc}




%------ using eurosym -------------------------------------------------------
\usepackage{eurosym}
\def\inh#1{\mbox{\small\euro}_{#1}}
\def\eith#1#2{\mathopen{[}#1 \ ,#2\mathclose{]}}

%------ using xy ------------------------------------------------------------
\usepackage[all]{xy}
%\def\larrow#1#2#3{\xymatrix{ #3 & #1 \ar[l] _-{#2} }}
\def\larrow#1#2#3{\xymatrix{ #3 & #1 \ar[l] _--{#2} }}
\def\rarrow#1#2#3{\xymatrix{ #1 \ar[r]^-{#2} & #3 }}
\def\arLaw#1#2#3#4#5{
\xymatrix{
        #1      \ar@/^1pc/[rr]^-{#4} &
        #5 &
        #2      \ar@/^1pc/[ll]^-{#3}
}}
\def\arLeq#1#2#3#4{\arLaw{#1}{#2}{#3}{#4}\leq}
%------ using pstricks (rnode etc) ------------------------------------------
%\usepackage{pstricks,pst-node,pst-text,pst-3d}
\input{macros/macros_drp}

\usepackage{macros/fitch}

\newcommand{\fitchr}[2]{\ensuremath{#1\text{\sc #2}}}

\newcommand{\conji}[2]{\ensuremath{\land\text{\textbf{\textsc{I}}}(#1,#2)}}
\newcommand{\conjel}[1]{\ensuremath{\land\text{\textbf{\textsc{E}}}_l(#1)}}
\newcommand{\conjer}[1]{\ensuremath{\land\text{\textbf{\textsc{E}}}_r(#1)}}

\newcommand{\disjil}[1]{\ensuremath{\lor\text{\textbf{\textsc{I}}}_l(#1)}}
\newcommand{\disjir}[1]{\ensuremath{\lor\text{\textbf{\textsc{I}}}_r(#1)}}
\newcommand{\disje}[3]{\ensuremath{\lor\text{\textbf{\textsc{E}}}(#1,#2,#3)}}

\newcommand{\negi}[1]{\ensuremath{\neg\text{\textbf{\textsc{I}}}(#1)}}
\newcommand{\nege}[1]{\ensuremath{\neg\text{\textbf{\textsc{E}}}(#1)}}

\newcommand{\falsei}[2]{\ensuremath{\bot\text{\textbf{\textsc{I}}}(#1,#2)}}
\newcommand{\falsee}[1]{\ensuremath{\bot\text{\textbf{\textsc{E}}}(#1)}}

\newcommand{\impi}[2]{\ensuremath{{\to}\text{\textbf{\textsc{I}}}(#1{-}#2)}}
\newcommand{\impe}[2]{\ensuremath{{\to}\text{\textbf{\textsc{E}}}(#1,#2)}}


\usepackage{macros/bussproofs}
\usepackage{bm}


\setLecture{2}{RAMDE -- Requirements and Model-driven Engineering}

% \title{
% 	RAMDE -- Requirements and Model-driven Engineering 
% 	}
% \author{David Pereira \and Jos\'{e} Proen\c{c}a}
% \institute{CISTER -- ISEP \\ Porto, Portugal}
% \date{MScCCSE 2020/21}


\begin{document}

\frame[plain]{\titlepage}


\section{First Order Logic and its Syntax}

\begin{slide}{The need for a richer kind of formal logic...}
\small

\begin{block}{The limitations of Propositional Logic}
 So far, we have been looking into Propositional Logic for reasoning about statements, in a way that can be valuable for the process of Requirement's Engineering. Although usefull, in most cases we need a richer language (and underlying formal system) that allows us to be more precise about the concepts we need to express.
\end{block}

\begin{block}{During this and next two classes...}
You will be presented with the concept of First Order Logic, learn about how can we express things using its language, learn how formulas can be evaluated with respect to models (yes, we are going to talk about models), and of course we will dive into performing Natural Deduction using First Order Logic constructions.
\end{block}


\begin{alert}{Warning:}
Things are going to get a little bit more complicated, considering what has been introduced in terms of Propositional Logic. Once again, bare with me and you will get comfortable with First Order Logic in a glimpse ;-)
\end{alert}

\end{slide}


\begin{slide}{First Order Logic - Syntax}

\begin{block}{Lets look into this simple example}
\begin{description}
  \item [Hypothesis 1:] All dogs like running.
  \item [Hypothesis 2:] Zen is a dog.
  \item [\color{red}{Conclusion:}] Zen likes to run.
\end{description}
\end{block}

\begin{block}{What can we say about the above reasoning?}
Well, the argument is clearly valid! However, translating it into propositional logic would result in a unique sentence $\varphi \land \psi \to \theta$ which is definitely not a valid formula!

Using truth tables and considering $\varphi$ = "All dogs like running", $\psi$ = "Zen is a dog", and $\theta$ = "Zen likes to run", if $f(\varphi) = {\bf true}$, $f(\psi) = {\bf true}$, and $f(\theta) = {\bf false}$ we would get that $f(\varphi \land \psi \to \theta) = {\bf false}$.
\end{block}
\end{slide}

\begin{slide}{First Order Logic - Syntax}

 \begin{block}{Representability of concepts in First Order Logic}
 In First Order Logic (FOL) we will be able to represent/reason about
   \begin{itemize}
     \item Objects
     \item Properties and relations about objects
     \item Properties and relations about sets of objects
   \end{itemize}
 \end{block}
 
 \begin{block}{Getting back to Zen's example}
 \begin{itemize}
   \item $\forall x, {\bf Dog}(x) \to {\bf LikesToRun}(x)$
   \item ${\bf Dog}({\tt zen})$
   \item ${\bf LikesToRun}({\tt zen})$
 \end{itemize}
  We will see further ahead in this and the following classes that this kind of reasoning is valid in FOL.
 \end{block}
\end{slide}


\begin{slide}{First Order Logic - Syntax}
 
  \begin{block}{FOL language}  
    A language of FOL considers the following sets of symbols:
    \begin{description}
    \item [logical symbols] of one of the following forms:
    \begin{itemize}
      \item a set of variables $S = \{x,y,\ldots,x_0,y_0,\ldots\}$  
      \item logical connectives $\land, \lor, \neg$, and $\to$
      \item quantifiers $\forall$ (for all) and $\exists$ (exists)
      \item parenthesis $($ and $)$
      \item possibly, the equality symbol $=$
    \end{itemize}
    \end{description}
  \end{block}

\end{slide}

\begin{slide}{First Order Logic - Syntax}  

  \begin{block}{FOL language}  
    A language of FOL considers the following sets of symbols:
    \begin{description}
    \item [Non-logical symbols] of one of the following forms:
      \begin{itemize}
      \item a (possibly empty) set of functional symbols for each $n$-arity, represented as $\mathcal{F}_n$ (when referring to constants, we are actually talking about functional symbols with arity $0$). Typically, $f$, $g$, $h$, $\ldots$
      \item a (possibly empty) set of relation symbols for each $n$-arity, represented as $\mathcal{R}_n$. Typically, $P$, $Q$, $R$, $\ldots$
      \end{itemize}
    \end{description}
  \end{block}
  



\end{slide}

\begin{slide}{First Order Logic - Syntax}
  \begin{block}{FOL Terms}  
  Let $\mathcal{L}$ be a FOL language. A term is inductively/recursively defined as follows:
  \begin{itemize}
  \item a variable $x \in \mathcal{V}$ is a term;
  \item a constant (i.e., a symbol $c \in \mathcal{F}_0$ is also a term;
  \item if $t_0,\ldots,t_n$ are terms and $f \in \mathcal{F}_n$ is a functional symbol, then $f(t_0,\ldots,t_n)$ is a term.  
  \end{itemize}
  \end{block}
  
  \begin{block}{Closed terms}
    A FOL term is said to be {\bf closed} if no variables occur in such term.
  \end{block}

\end{slide}

\begin{slide}{Some quick examples}

Assuming that $\mathcal{F}_0 = \{a,d\}$, that $\mathcal{F}_1 = \{f\}$, that $\mathcal{F}_2 = \{h\}$, and that $\mathcal{F}_3 = \{g\}$

\begin{itemize}
\item \color<2->{red}{$f(a,g(x,g(a),a))$}
\item \structure<2->{$h(d,h(f(a),x))$}
\item \color<2->{red}{$x(d,g(y))$}
\item \structure<2->{$h(h(x,x),h(y,y))$}
\item \color<2->{red}{$f(a(x))$}
\end{itemize}

\end{slide}

\begin{slide}{First Order Logic - Syntax}
  \begin{block}{FOL Atoms}  
  Let $\mathcal{L}$ be a FOL language. An {\bf atom} (from the term atomic formula) is inductively/recursively defined as follows:
  \begin{itemize}
  \item if $t_0,\ldots,t_n$ are terms and $R \in \mathcal{R}_n$ is a relational symbol, then $R(t_0,\ldots,t_n)$ is an atom;
  \item if $\mathcal{L}$ include the equality symbol $=$ and if $t_1$ and $t_2$ are terms, then $t_1 = t_2$ is an atom.  
  \end{itemize}
  \end{block}
  
  \begin{block}{Some examples...}
  \begin{columns}
    \begin{column}{0.48\textwidth}
      \begin{itemize}
        \item $R(b)$
        \item $R(x,y,z)$
        \item $G(f(a,b),x)$
      \end{itemize}
    \end{column}
    \begin{column}{0.48\textwidth}
      \begin{itemize}
        \item $R(f(a,x),g(y,b),h(c))$
      \end{itemize}        
    \end{column}
  \end{columns}
  \end{block}

\end{slide}

\begin{slide}{First Order Logic - Syntax}
  \begin{block}{FOL Formulae}  
  Let $\mathcal{L}$ be a FOL language. The set of {\bf formulae} is inductively/recursively defined  as follows:
  \begin{itemize}
  \item an atom is a formula;
  \item if $\varphi$ is a formula, then so is $\neg\varphi$;
  \item if $\varphi$ and $\psi$ are formulas, then so are $\varphi \land \psi$, $\varphi \lor \psi$, and $\varphi \to \psi$;
  \item if $\varphi$ is a formula and $x$ is a variable, then $\forall x, \varphi$ and $\exists x, \varphi$ are also formulas.
  \end{itemize}
  \end{block}
\end{slide}

\begin{slide}{Some quick examples}

Assuming that $\mathcal{F}_0 = \{a,d\}$, that $\mathcal{F}_1 = \{f\}$, that $\mathcal{F}_2 = \{h\}$, and that $\mathcal{R}_2 = \{R,S\}$, which of the following expressions are formulas?

\begin{itemize}
\item \structure<2->{$R(a,d)$}
\item \color<2->{red}{$h(x,y)$}
\item \color<2->{red}{$S(R(f(x),y),z)$}
\item \structure<2->{$R(d,a) \to \exists y, S(d,y)$}
\item \color<2->{red}{$S(\forall x, R(f(a),d),x)$}
\item \structure<2->{$\forall x, R(f(a),h(a,x))$}
\end{itemize}

\end{slide}

\begin{slide}{A more serious example}
  Lets look at a FOL that focus on being able to express formulae about natural numbers, including addition and multiplication. Let $\mathcal{A}$ be a language that includes equality and such that $\mathcal{F}_0 = \{0,1\}$, $\mathcal{F}_2 = \{+,\times\}$, and $\mathcal{R}_2 = \{<\}$.
  
  Terms of this language include:
  \begin{itemize}
  \item $0$, $1$, $+(1,1)$, $+(1,+(1,1))$, $+(1,+(+(1,1),1))$
  \item $\times(+(1,0),1)$, $\times(\times(1,1),+(0,1))$  
  \end{itemize}
  
  Formulas of this language include:
  \begin{itemize}
  \item $<(\times(1,1),+(1,1))$
  \item $\forall x, (+(0,x) = x)$
  \item $\forall x, \exists y, +(x,1) = +(y,1) \to x = y$  
  \end{itemize}


\end{slide}

\begin{slide}{Translating text into FOL formulae}
\begin{block}{In general, we have the following translation}
\begin{center}
\begin{tabular}{c|c}
Type of text &  FOL formula \\ \hline
 All $P$ are also $Q$ & $\forall x, (P(x) \to Q(x))$ \\
 Some $P$ are also $Q$ & $\exists x, (P(x) \land Q(x))$ \\
 No $P$ is $Q$ & $\forall x, (P(x) \to \neg Q(x))$ \\
 Not all $P$ are $Q$ & $\exists x, (P(x) \land \neg Q(x))$
\end{tabular}
\end{center}
\end{block}
\end{slide}

\begin{slide}{Small exercise}
Find some adequate predicates to express the following statements:
\begin{itemize}
\item All even numbers are prime numbers\\
\onslide<2->{$\forall x, Even(x) \to Prime(x)$}
\item Not all prime numbers are even numbers\\
\onslide<3->{$\neg(\forall x, Prime(x) \to Even(x)) $}
\item Some prime numbers are not even\\
\onslide<4->{$\exists x, Prime(x) \land \neg Even(x)$}
\item All prime number is not even or equal to 2\\  
\onslide<5->{$\forall x, Prime(x) \to (\neg Even(x) \lor x = 2)$}
\end{itemize}
\end{slide}

\begin{slide}{Bound and free variables}
  \begin{block}{Bound Variable}
  A variable $x$ is said to be {\bf bound} to a formula $\varphi$ if $\varphi$ has a subformula $\psi$ whose schema is $\forall x, \theta$ or $\exists x, \theta$ and $x$ occurs in $\theta$. 
  \end{block}
  \begin{block}{Free Variable}
  A variable $x$ is said to be {\bf free} if it is not bound.  
  \end{block}
  \begin{block}{Proposition}
  A formula is said to be a proposition if it does not contain free variables.  
  \end{block}
\end{slide}

\begin{slide}{Small exercises}
For each of the formulae presented bellow, identify the bound and free variables:
\begin{itemize}
\item $\exists x, (P(y,z) \land \forall y, (\neg Q(x,y) \lor P(y,z))$  
\item $\neg(\forall x, \exists y, P(x,y,z) \to \forall z, P(x,y,z))$
\item $P(a,g(c,d))$
\item $\exists x, (P(x) \to \neg Q(x))$
\end{itemize}
\end{slide}

\section*{Semantics}

\begin{slide}{How to Evaluate a FOL Formula}
  \begin{block}{Recalling evaluation of formulae in Propositional Logic}
  In PL, the evaluation of a formula $\varphi$ is fully determined by the valuation given to each of its propositional variables, and the connectives involved in the formula.  
  \end{block}
  
  \begin{block}{But how is it done in FOL?}
  In FOL, to evaluate a formula we need to determine the meaning of:
    \begin{itemize}
    \item bound and free variables
    \item quantifiers
    \item functional symbols
    \item relational/predicate symbols
    \end{itemize}
    This must be done in a concrete {\bf universe}.
  \end{block}
\end{slide}

%\begin{slide}{}
%  
%\end{slide}
%
%\begin{slide}{}
%  
%\end{slide}
%
%\begin{slide}{}
%  
%\end{slide}
%
%\begin{slide}{}
%  
%\end{slide}
%
%\begin{slide}{}
%  
%\end{slide}
%
%\begin{slide}{}
%  
%\end{slide}
%
%\begin{slide}{}
%  
%\end{slide}
%
%\begin{slide}{}
%  
%\end{slide}
%
%\begin{slide}{}
%  
%\end{slide}
%
%\begin{slide}{}
%  
%\end{slide}
%
%\begin{slide}{}
%  
%\end{slide}
%
%\begin{slide}{}
%  
%\end{slide}
%
%\begin{slide}{}
%  
%\end{slide}
%
%\begin{slide}{}
%  
%\end{slide}
%
%\begin{slide}{}
%  
%\end{slide}
%
%\begin{slide}{}
%  
%\end{slide}
%
%\begin{slide}{}
%  
%\end{slide}
%
%\begin{slide}{}
%  
%\end{slide}
%
%\begin{slide}{}
%  
%\end{slide}
%
%\begin{slide}{}
%  
%\end{slide}
%
%\begin{slide}{}
%  
%\end{slide}




\end{document}
