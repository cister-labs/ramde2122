\documentclass[aspectratio=169]{beamer}
\usepackage{etex} % fixes new-dimension error

\input{macros/beamerconf}
\usepackage{etex} % fixes new-dimension error
\usepackage{lmodern} % fixes warnings
\usepackage{textcomp}% fixes warnings

\usepackage{graphicx,amsmath}
\usepackage{stmaryrd} % cf. interleave
%\usepackage{./macros/myisolatin1}
%\usepackage{./macros/prooftree}
\usepackage{alltt}
%\usepackage{./macros/circle}
\usepackage{listings}
\usepackage{relsize} % relative size fonts
\usepackage[normalem]{ulem} % strikethrough text (with \sout{.})
\usepackage{tikz}
\usetikzlibrary{%
  positioning
 ,patterns
 ,arrows
 ,automata
 ,calc
 ,shapes
 ,fit
 ,fadings
 ,decorations.pathreplacing
 ,plotmarks
% ,pgfplots.groupplots
 ,decorations.markings
}
% \tikzset{shorten >=1pt,node distance=2cm,on grid,auto,initial text={},inner sep=2pt}
\tikzstyle{aut}=[shorten >=1pt,node distance=2cm,on grid,auto,initial text={},inner sep=2pt]
\tikzstyle{st}=[circle,draw=black,fill=black!10,inner sep=3pt]
\tikzstyle{sst}=[rectangle,draw=none,fill=none,inner sep=3pt]
\tikzstyle{final}=[accepting]
\usepackage[normalem]{ulem} % striking out text with \sout{...}
\usepackage{xspace}

\usepackage{transparent}

% Nicer TT fonts
\usepackage[scaled=.83]{beramono}
\usepackage[T1]{fontenc}




%------ using eurosym -------------------------------------------------------
\usepackage{eurosym}
\def\inh#1{\mbox{\small\euro}_{#1}}
\def\eith#1#2{\mathopen{[}#1 \ ,#2\mathclose{]}}

%------ using xy ------------------------------------------------------------
\usepackage[all]{xy}
%\def\larrow#1#2#3{\xymatrix{ #3 & #1 \ar[l] _-{#2} }}
\def\larrow#1#2#3{\xymatrix{ #3 & #1 \ar[l] _--{#2} }}
\def\rarrow#1#2#3{\xymatrix{ #1 \ar[r]^-{#2} & #3 }}
\def\arLaw#1#2#3#4#5{
\xymatrix{
        #1      \ar@/^1pc/[rr]^-{#4} &
        #5 &
        #2      \ar@/^1pc/[ll]^-{#3}
}}
\def\arLeq#1#2#3#4{\arLaw{#1}{#2}{#3}{#4}\leq}
%------ using pstricks (rnode etc) ------------------------------------------
%\usepackage{pstricks,pst-node,pst-text,pst-3d}
\input{macros/macros}


\setLecture{11}{More on Requirements: The EARS approach and the Doorstop tool}

\begin{document}

\frame[plain]{\titlepage}

\section*{The EARS Approach to Requirements Specification}

\begin{frame}
  \small
  \frametitle{Getting to Know EARS}
  \begin{block}{WHAT IS EARS?}
  \begin{itemize}
    \item The acronym {\bf EARS} stands for {\it "Easy Approach to Requirements Syntax"} 
    \item EARS is a mechanism to gently constrain textual requirements
    \item EARS patterns provide structured guidance that enable authors to write high quality textual requirements.
  \end{itemize}
\end{block}

\begin{block}{EARS Building Blocks}
  \begin{itemize}
    \item There is a set syntax (structure), with an underlying ruleset. 
    \item A small number of keywords are used to denote the different clauses of an EARS requirement. 
    \item The clauses are always in the same order, following temporal logic. 
    \item The syntax and the keywords closely match common usage of English and are therefore intuitive.
  \end{itemize}  
\end{block}

\end{frame}


\begin{frame}
  \small
  \frametitle{How EARS Came to Life}
  \begin{block}{Context}
  \begin{itemize}
    \item Ideas triggered while the author and colleagues at Rolls-Royce PLC were analysing the airworthiness regulations for an engine jet control system. 
    \item The regulations contained high level objectives, a mixture of implicit and explicit requirements at different levels, lists, guidelines and supporting information.
    \item In the process of extracting and simplifying the requirements, Mav noticed that the requirements all followed a similar structure.
    \item He found that requirements were easiest to read when the clauses always appeared in the same order. These patterns were refined and evolved to create EARS.
    \item The notation was first published in 2009 and has been adopted by many organisations across the world.
  \end{itemize}
  \end{block}
  
\end{frame}

\begin{frame}
  \frametitle{Motivations for Adopting EARS (1/2)}
  \begin{block}{Why adopt EARS?}
  \begin{itemize}
  \item  System requirements are usually written in unconstrained natural language, which is inherently imprecise. 
  \item Often, requirements authors are not trained in how to write requirements. 
  \item During system development, requirements problems propagate to lower levels. 
  \end{itemize}
  {\bf This creates unnecessary volatility and risk, impacting programme schedule and cost.}
  \end{block} 
\end{frame}

\begin{frame}
  \frametitle{Motivations for Adopting EARS (2/2)}
  \begin{block}{Why adopt EARS?}
  \begin{itemize}
    \item EARS reduces or even eliminates common problems found in natural language requirements. 
    \item It is especially effective for requirements authors who must write requirements in English, but whose first language is not English. 
    \item EARS has proved popular with practitioners because it is lightweight, there is little training overhead, no specialist tool is necessary, and the resultant requirements are easy to read.
  \end{itemize}
  \end{block} 
\end{frame}

\begin{frame}
  \frametitle{EARS Cont...}
  \begin{block}{WHO USES EARS?}
    EARS is used worldwide by large and small organisations in different domains. These include blue chip companies such as Bosch, Honeywell, Intel, Rolls-Royce and Siemens.

The notation is taught at universities around the world including in China, France, Sweden, UK and USA.
  \end{block}

\end{frame}

\section*{The EARS Patterns}

\begin{frame}[fragile]
  \frametitle{THE EARS PATTERNS}
  \begin{block}{Generic EARS syntax}
    The clauses of a requirement written in EARS always appear in the same order. The basic structure of an EARS requirement is:
    \begin{verbatim}
    While <optional pre-condition>, 
     when <optional trigger>, 
      the <system name> shall <system response>  
    \end{verbatim}
  The EARS ruleset states that a requirement must have: Zero or many preconditions; Zero or one trigger; One system name; One or many system responses.

The application of the EARS notation produces requirements in a small number of patterns, depending on the clauses that are used. The patterns are illustrated below.
  \end{block}
  

\end{frame}

\begin{frame}[fragile]
  \frametitle{The EARS Patterns}
  \begin{block}{Ubiquitous requirements}
    These refer to requirements that are always active. No EARS specific keyword is present when specifying this particular type of requirement.
    
    The pattern for 
    \begin{verbatim}
    The <system name> shall <system response>  
    \end{verbatim}
  \end{block}
  \begin{example}
    \begin{itemize}
      \item The mobile phone shall have a mass of less than XX grams.
      \item The kitchen system shall have an input hatch
    \end{itemize}
  \end{example}
\end{frame}


\begin{frame}[fragile]
  \frametitle{THE EARS PATTERNS}
  \begin{block}{State driven requirements}
  State driven requirements are active as long as the specified state remains true and are denoted by the keyword While.
    \begin{verbatim}
      While <precondition(s)>, 
       the <system name> 
        shall <system response>
    \end{verbatim}  
  \end{block}
  
  \begin{example}
  \begin{itemize}
    \item While there is no card in the ATM, the ATM shall display “insert card to begin.
    \item While the kitchen system is in maintenance mode, the kitchen system shall reject all input.
  \end{itemize}
  \end{example}

\end{frame}

\begin{frame}[fragile]
  \frametitle{THE EARS PATTERNS}
  \begin{block}{Event driven requirements}
  Event driven requirements specify how a system must respond when a triggering event occurs and are denoted by the keyword When.
    \begin{verbatim}
      When <trigger>, 
        the <system name> 
         shall <system response>
    \end{verbatim}
  \end{block}
  
  \begin{example}
  \begin{itemize}
    \item When mute is selected, the laptop shall suppress all audio output.
    \item When the chef inserts potato to the input hatch, the kitchen system shall peel the potato.
  \end{itemize}
  \end{example}
\end{frame}



\begin{frame}[fragile]
  \frametitle{THE EARS PATTERNS}
  \begin{block}{Optional feature requirements}
  Optional feature requirements apply in products or systems that include the specified feature and are denoted by the keyword Where.
    \begin{verbatim}
      Where <feature is included>, 
       the <system name> 
        shall <system response>
    \end{verbatim}
  \end{block}
  
  \begin{example}
  \begin{itemize}
    \item Where the car has a sunroof, the car shall have a sunroof control panel on the driver door.
    \item Where the kitchen system has a food freshness sensor, the kitchen system shall detect rotten foodstuffs. 
  \end{itemize}
  \end{example}
\end{frame}



\begin{frame}[fragile]
  \frametitle{THE EARS PATTERNS}
  \begin{block}{Unwanted behaviour requirements}
    These are used to specify the required system response to undesired situations and are denoted by the keywords If and Then.
    \begin{verbatim}
     If <trigger>, 
      then the <system name> 
       shall <system response> 
    \end{verbatim}
      \end{block}
  
  \begin{example}
  \begin{itemize}
    \item If an invalid credit card number is entered, then the website shall display "please re-enter credit card details"
    \item  If a spoon is inserted to the input hatch, then the kitchen system shall eject the spoon.
  \end{itemize}
  \end{example}
\end{frame}



\begin{frame}[fragile]
  \frametitle{THE EARS PATTERNS}
  \begin{block}{Complex requirements}
  The simple building blocks of the EARS patterns described above can be combined to specify requirements for richer system behaviour. Requirements that include more than one EARS keyword are called Complex requirements.
    \begin{verbatim}
     While <precondition(s)>, 
      When <trigger>, 
       the <system name> shall <system response> 
    \end{verbatim}
    Example: While the aircraft is on ground, when reverse thrust is commanded, the engine control system shall enable reverse thrust.

Complex requirements for unwanted behaviour also include the If-Then keywords.
  \end{block}

\end{frame}

\section*{How to apply EARS patterns}

\begin{frame}[fragile]
  \frametitle{THE EARS PATTERNS}
  \begin{itemize}
  \item Identify whether you are working with a requirement, or something else (e.g., note, example, remark, etc)
  \item Identify compound requirements, i.e., whether the requirement needs to be split/decomposed 
  \item Identify the acting system, person, or process
  \item Analise the needed sentence type(s)
  \item Identigy possible missing requirements
  \item Analyse the translated requirements for ambiguity, conflict, and repetition
  \item Review requirements if possible
  \item Interate as required
  \end{itemize}

\end{frame}

\section*{Troubleshooting EARS problems}

\begin{frame}
  \frametitle{What are the issues that can occur when using EARS?}
  \begin{itemize}
  \item {\it No sentence type fits:} Maybe you are not translating a requirement?
  \item {\it Can't identify the actor:} either use higher abstraction level until it makes sense, or get more information from the relevant stakeholder
  \item {\it There is no system response:}  typically the case with non-functional requirements; it can be expressed as "the system shall be ..."
  \item {\it There is no template for "shall not":} try using "shall be immune" or similar or, as last resort, use the "shall not" pattern
  \item {\it EARS produces too many atomic requirements:} 
  \end{itemize}
  
\end{frame}





\end{document}
