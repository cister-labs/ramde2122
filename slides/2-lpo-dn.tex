\documentclass[aspectratio=169]{beamer}

\input{macros/beamerconf}
\usepackage{etex} % fixes new-dimension error
\usepackage{lmodern} % fixes warnings
\usepackage{textcomp}% fixes warnings

\usepackage{graphicx,amsmath}
\usepackage{stmaryrd} % cf. interleave
%\usepackage{./macros/myisolatin1}
%\usepackage{./macros/prooftree}
\usepackage{alltt}
%\usepackage{./macros/circle}
\usepackage{listings}
\usepackage{relsize} % relative size fonts
\usepackage[normalem]{ulem} % strikethrough text (with \sout{.})
\usepackage{tikz}
\usetikzlibrary{%
  positioning
 ,patterns
 ,arrows
 ,automata
 ,calc
 ,shapes
 ,fit
 ,fadings
 ,decorations.pathreplacing
 ,plotmarks
% ,pgfplots.groupplots
 ,decorations.markings
}
% \tikzset{shorten >=1pt,node distance=2cm,on grid,auto,initial text={},inner sep=2pt}
\tikzstyle{aut}=[shorten >=1pt,node distance=2cm,on grid,auto,initial text={},inner sep=2pt]
\tikzstyle{st}=[circle,draw=black,fill=black!10,inner sep=3pt]
\tikzstyle{sst}=[rectangle,draw=none,fill=none,inner sep=3pt]
\tikzstyle{final}=[accepting]
\usepackage[normalem]{ulem} % striking out text with \sout{...}
\usepackage{xspace}

\usepackage{transparent}

% Nicer TT fonts
\usepackage[scaled=.83]{beramono}
\usepackage[T1]{fontenc}




%------ using eurosym -------------------------------------------------------
\usepackage{eurosym}
\def\inh#1{\mbox{\small\euro}_{#1}}
\def\eith#1#2{\mathopen{[}#1 \ ,#2\mathclose{]}}

%------ using xy ------------------------------------------------------------
\usepackage[all]{xy}
%\def\larrow#1#2#3{\xymatrix{ #3 & #1 \ar[l] _-{#2} }}
\def\larrow#1#2#3{\xymatrix{ #3 & #1 \ar[l] _--{#2} }}
\def\rarrow#1#2#3{\xymatrix{ #1 \ar[r]^-{#2} & #3 }}
\def\arLaw#1#2#3#4#5{
\xymatrix{
        #1      \ar@/^1pc/[rr]^-{#4} &
        #5 &
        #2      \ar@/^1pc/[ll]^-{#3}
}}
\def\arLeq#1#2#3#4{\arLaw{#1}{#2}{#3}{#4}\leq}
%------ using pstricks (rnode etc) ------------------------------------------
%\usepackage{pstricks,pst-node,pst-text,pst-3d}
\input{macros/macros_drp}

\usepackage{macros/fitch}

\newcommand{\fitchr}[2]{\ensuremath{#1\text{\sc #2}}}

\newcommand{\conji}[2]{\ensuremath{\land\text{\textbf{\textsc{I}}}(#1,#2)}}
\newcommand{\conjel}[1]{\ensuremath{\land\text{\textbf{\textsc{E}}}_l(#1)}}
\newcommand{\conjer}[1]{\ensuremath{\land\text{\textbf{\textsc{E}}}_r(#1)}}

\newcommand{\disjil}[1]{\ensuremath{\lor\text{\textbf{\textsc{I}}}_l(#1)}}
\newcommand{\disjir}[1]{\ensuremath{\lor\text{\textbf{\textsc{I}}}_r(#1)}}
\newcommand{\disje}[3]{\ensuremath{\lor\text{\textbf{\textsc{E}}}(#1,#2,#3)}}

\newcommand{\negi}[1]{\ensuremath{\neg\text{\textbf{\textsc{I}}}(#1)}}
\newcommand{\nege}[1]{\ensuremath{\neg\text{\textbf{\textsc{E}}}(#1)}}

\newcommand{\falsei}[2]{\ensuremath{\bot\text{\textbf{\textsc{I}}}(#1,#2)}}
\newcommand{\falsee}[1]{\ensuremath{\bot\text{\textbf{\textsc{E}}}(#1)}}

\newcommand{\impi}[2]{\ensuremath{{\to}\text{\textbf{\textsc{I}}}(#1{-}#2)}}
\newcommand{\impe}[2]{\ensuremath{{\to}\text{\textbf{\textsc{E}}}(#1,#2)}}

\newcommand{\eqi}{\ensuremath{{=}\text{\textbf{\textsc{I}}}}}
\newcommand{\eqe}[3]{\ensuremath{{=}\text{\textbf{\textsc{E}}}(#1,#2,#3)}}

\newcommand{\foralli}[2]{\ensuremath{{\forall}\text{\textbf{\textsc{I}}}}(#1{-}#2)}
\newcommand{\foralle}[1]{\ensuremath{{\forall}\text{\textbf{\textsc{E}}}(#1)}}

\newcommand{\exi}[1]{\ensuremath{{\exists}\text{\textbf{\textsc{I}}}}(#1)}
\newcommand{\exe}[2]{\ensuremath{{\exists}\text{\textbf{\textsc{E}}}(#1{-}#2)}}

\newcommand{\fas}{\;\,\,\,\fa}
\newcommand{\fal}[1]{{#1}\,\,\fa}


\usepackage{macros/bussproofs}
\usepackage{bm}


\setLecture{2}{RAMDE -- Requirements and Model-driven Engineering}

% \title{
% 	RAMDE -- Requirements and Model-driven Engineering 
% 	}
% \author{David Pereira \and Jos\'{e} Proen\c{c}a}
% \institute{CISTER -- ISEP \\ Porto, Portugal}
% \date{MScCCSE 2020/21}


\begin{document}

\frame[plain]{\titlepage}


\section{Natural Deduction in First Order Logic}

\begin{slide}{Recalling FOL...}

  \begin{block}{FOL language}  
    A language of FOL considers the following sets of symbols:
    \begin{description}
    \item [logical symbols] of one of the following forms: a \textbf{set of variables} $S = \{x,y,\ldots,x_0,y_0,\ldots\}$; \textbf{logical connectives} $\land, \lor, \neg$, and $\to$; \textbf{quantifiers} $\forall$ (for all) and $\exists$ (exists); parenthesis $($ and $)$; possibly, the \textbf{equality} symbol $=$
    \item [Non-logical symbols] of one of the following forms: a (possibly empty) set of \textbf{functional symbols} for each $n$-arity, represented as $\mathcal{F}_n$ (when referring to constants, we are actually talking about functional symbols with arity $0$). Typically, $f$, $g$, $h$, $\ldots$; a (possibly empty) set of \textbf{relation symbols} for each $n$-arity, represented as $\mathcal{R}_n$. Typically, $P$, $Q$, $R$, $\ldots$
    \end{description}
  \end{block}
\end{slide}

\begin{frame}[shrink=0.95]{First Order Logic - Syntax}
  \begin{block}{FOL Terms and Atoms}  
  Let $\mathcal{L}$ be a FOL language. A term is inductively/recursively defined as follows:
  \begin{itemize}
  \item a variable $x \in \mathcal{V}$ is a term;
  \item a constant i.e., a symbol $c \in \mathcal{F}_0$ is also a term;
  \item if $t_0,\ldots,t_n$ are terms and $f \in \mathcal{F}_n$ is a functional symbol, then $f(t_0,\ldots,t_n)$ is a term.  
  \end{itemize}
  A FOL term is said to be {\bf closed} if no variables occur in such term.\\
  \vspace{5pt}
  Let $\mathcal{L}$ be a FOL language. An {\bf atom} (from the term atomic formula) is inductively/recursively defined as follows:
  \begin{itemize}
  \item if $t_0,\ldots,t_n$ are terms and $R \in \mathcal{R}_n$ is a relational symbol, then $R(t_0,\ldots,t_n)$ is an atom;
  \item if $\mathcal{L}$ include the equality symbol $=$ and if $t_1$ and $t_2$ are terms, then $t_1 = t_2$ is an atom.  
  \end{itemize}
  \end{block}
\end{frame}


\begin{slide}{First Order Logic - Syntax}
 
 \begin{block}{FOL Formulae}  
  Let $\mathcal{L}$ be a FOL language. The set of {\bf formulae} is inductively/recursively defined  as follows:
  \begin{itemize}
  \item an atom is a formula;
  \item if $\varphi$ is a formula, then so is $\neg\varphi$;
  \item if $\varphi$ and $\psi$ are formulas, then so are $\varphi \land \psi$, $\varphi \lor \psi$, and $\varphi \to \psi$;
  \item if $\varphi$ is a formula and $x$ is a variable, then $\forall x, \varphi$ and $\exists x, \varphi$ are also formulas.
  \end{itemize}
  \end{block}

\end{slide}



\begin{slide}{Bound and free variables}
  \begin{block}{Bound Variable}
  A variable $x$ is said to be {\bf bound} to a formula $\varphi$ if $\varphi$ has a subformula $\psi$ whose schema is $\forall x, \theta$ or $\exists x, \theta$ and $x$ occurs in $\theta$. 
  \end{block}
  \begin{block}{Free Variable}
  A variable $x$ is said to be {\bf free} if it is not bound.  
  \end{block}
  \begin{block}{Proposition}
  A formula is said to be a proposition if it does not contain free variables.  
  \end{block}
\end{slide}



\begin{slide}{Variable Substitution}

\begin{block}{Substitution}
Let $\mathcal{L}$ be a FOL language, $\varphi$ a formula, $t$ a term, and $x \in \mathcal{L}$ a variable. The substitution of the variable $x$ by the term $t$ in $\varphi$ is denoted by $\varphi[t/x]$ and corresponds to replacing all the free occurrences of $x$ in $\varphi$ by the term $t$.
\end{block}

\end{slide}


\section*{Natural Deduction Rules}

\begin{frame}[shrink=0.95]{Which are the new rules (on top of Propositional Logic)?}
  \begin{block}{Elimination rule for $\forall$}
  If we know that $\forall x, \varphi$ holds, then we can conclude that $\varphi$ holds for a specific term $t$
  \begin{prooftree}
      \AxiomC{$\forall x\,\varphi$}
      \RightLabel{$\forall\text{\textbf{\textsc{E}}}$}
      \UnaryInfC{$\varphi[t/x]$}
    \end{prooftree}  
  \end{block}
  
  \begin{block}{Introduction rule for $\forall$}
  If we assume some term $t$ and we are able to prove that $\varphi[t/x]$ then we can conclude that $\forall x, \varphi$.
  \begin{prooftree}
      \AxiomC{$[t]$}
        \noLine
        \UnaryInfC{$$}
        \noLine
        \UnaryInfC{$$}
        \noLine
        \UnaryInfC{$$}
        \noLine
        \UnaryInfC{$$}

      \AxiomC{$$}
        \noLine
        \UnaryInfC{$$}
        \noLine
        \UnaryInfC{$\vdots$}
        \noLine
        \UnaryInfC{$$}
        \noLine
        \UnaryInfC{$$}

        
      \AxiomC{$$}
        \noLine
        \UnaryInfC{$$}
        \noLine
        \UnaryInfC{$$}
        \noLine
        \UnaryInfC{$$}
        \noLine
        \UnaryInfC{$\varphi[t/x]$}

      \RightLabel{$\forall\text{\textbf{\textsc{I}}}$}
      
      \TrinaryInfC{$\forall x\,\varphi[t/x]$}
    \end{prooftree}
  \end{block}
  
\end{frame}

\begin{slide}{Examples of reasoning about "for all"}
  
Lets prove that $\forall x, (P(x) \to Q(x)), \forall x, P(x) \vdash \forall x, Q(x)$.\vspace{0.5cm}
%\setlength{\fitchindent}{1em}
\begin{fitch}
    \fa \forall x, (P(x) \to Q(x)) \\
    \fj \forall x, P(x) \\
    \ftag{~}{\fa} \\
    \fa \fal{t} P(t) \to Q(t) & \foralle{1}\\
    \fa \fas P(t) & \foralle{2} \\
    \fa \fas Q(t) & \impe{3}{4}\\
    \fa \forall x, Q(x) & \foralli{3}{5}
\end{fitch}
\end{slide}

\begin{slide}{Examples of reasoning about "for all"}
  
Lets prove that $P(t),\forall x (P(x) \to Q(x)) \vdash \neg Q(t)$.\vspace{0.5cm}
%\setlength{\fitchindent}{1em}
\begin{fitch}
    \fa P(t) \\
    \fj \forall x, (P(x) \to Q(x)) \\
    \fa P(t) \to Q(t) & \foralle{2}\\
    \fa \neg Q(t) & \impe{3}{1}
\end{fitch}
\end{slide}

\begin{slide}{Examples of reasoning about "for all"}
Lets prove that $\vdash \forall x (P(x) \to Q(x)) \to (\forall x, P(x) \to \forall x, Q(x))$.\vspace{0.5cm}
%\setlength{\fitchindent}{1em}
\begin{fitch}
    \fa \fj \forall x (P(x) \to Q(x)) \\
    \fa \fa \fj \forall x P(x) \\
    \fa \fa \fa t\,\,\fa P(t) & \foralle{2}\\
    \fa \fa \fa \;\,\,\,\fa P(t) \to Q(t) & \foralle{1} \\
    \fa \fa \fa \;\,\,\,\fa Q(t) & \impe{3}{4} \\
    \fa \fa \fa \forall x Q(x) & \foralli{3}{5} \\
    \fa \fa \forall x P(x) \to \forall x Q(x) & \impi{2}{6} \\
    \fa \forall x (P(x) \to Q(x)) \to (\forall x, P(x) \to \forall x, Q(x)) & \impi{1}{7}
\end{fitch}
\end{slide}

\begin{frame}[shrink=0.95]{Which are the new rules (on top of Propositional Logic)?}
  \begin{block}{Elimination rule for $\exists$}
  If we know that $\exists x, \varphi$ holds, and if assuming term $t$ and $\varphi[t/x]$ we can deduce $\psi$, then we can prove $\psi$ overall.
  \begin{prooftree}
      \AxiomC{$\exists x\,\varphi$}
      \AxiomC{$[t\:\:\:\varphi[t/x]]$}
      \noLine
      \UnaryInfC{$\vdots$}
      \noLine
      \UnaryInfC{$\psi$}
      \RightLabel{$\exists\text{\textbf{\textsc{E}}}$}
      \BinaryInfC{$\psi$}
    \end{prooftree}  
  \end{block}
  
  \begin{block}{Introduction rule for $\exists$}
  If we assume some term $t$ and we are able to prove that $\varphi[t/x]$ then we can conclude that $\forall x, \varphi$.
  \begin{prooftree}
      \AxiomC{$\varphi[t/x]$}
      \RightLabel{$\exists\text{\textbf{\textsc{I}}}$}
      \UnaryInfC{$\exists x, \varphi$}  
  \end{prooftree}
  \end{block}
  
\end{frame}



\begin{slide}{Examples of reasoning about "for all"}
Lets prove that $\forall x, \varphi \vdash \exists x, \varphi$.\\\vspace{0.5cm}
\begin{fitch}
    \fj \forall x \varphi \\
    \fa \varphi[t/x] & \foralle{1}\\
    \fa \exists x, \varphi & \exi{2}
\end{fitch}
\end{slide}

\begin{slide}{Examples of reasoning about "for all"}
Lets prove that $\forall x (P(x) \to Q(x)), \exists x P(x) \vdash \exists x Q(x)$.\\\vspace{0.5cm}
\begin{fitch}
    \fa \forall x P(x) \to Q(x) \\
    \fj \exists x Q(x) \\
    \fa t\,\,\fj P(t) \\
    \fa \;\,\,\,\fa P(t) \to Q(t) & \foralle{1} \\
    \fa \;\,\,\,\fa Q(t) & \impe{3}{4} \\
    \fa \;\,\,\,\fa \exists x Q(x) & \exi{5}\\
    \fa \exists x Q(x) & \exe{3}{6}
\end{fitch}
\end{slide}

\begin{frame}[shrink=10]
\frametitle{Examples of reasoning about "for all"}
Lets prove that $\exists x P(x), \forall x\forall y (P(x) \to Q(y)) \vdash \forall y Q(y)$.\\\vspace{0.5cm}
\begin{fitch}
    \fa \exists x Q(x) \\
    \fj \forall x \forall y (P(x) \to Q(y)) \\
    \ftag{~}{\fa} \\
    \fa t\,\,\fa u\,\,\fj P(t) \\
    \fa \;\,\,\,\fa\,\,\,\,\,\fa \forall y (P(t) \to Q(y)) & \foralle{2}   \\
    \fa \;\,\,\,\fa\,\,\,\,\,\fa P(t) \to Q(u) & \foralle{4}   \\
    \fa \;\,\,\,\fa\,\,\,\,\,\fa Q(u) & \impe{3}{5} \\
    \fa \;\,\,\,\fa Q(u) & \exe{1}{3-6} \\
    \fa \forall y Q(y) & \foralli{3}{7}  \\
\end{fitch}
\end{frame}


\begin{frame}[shrink=0.95]{Which are the new rules (on top of Propositional Logic)?}
  \begin{block}{Elimination rule for $=$}
  If we know that two terms $t_1$ and $t_2$ are equal and that $\varphi[t_1/x]$ holds, then $\varphi[t_1/x]$ must also hold.
  \begin{prooftree}
      \AxiomC{$t_1 = t_2$}
      \AxiomC{$\varphi[t_1/x]$}
      \RightLabel{${=}\text{\textbf{\textsc{E}}}$}
      \BinaryInfC{$\varphi[t_2/x]$}
    \end{prooftree}  
  \end{block}
  
  \begin{block}{Introduction rule for $=$}
  If we assume some term $t$ and we are able to prove that $\varphi[t/x]$ then we can conclude that $\forall x, \varphi$.
  \begin{prooftree}
      \AxiomC{$$}
      \RightLabel{${=}\text{\textbf{\textsc{I}}}$}
      \UnaryInfC{$t = t$}  
  \end{prooftree}
  \end{block}
  
\end{frame}

\begin{slide}{Examples of reasoning about equality}
  \begin{columns}
      \begin{column}{0.48\textwidth}
      Lets prove that if $t_1 = t_2$ then $t_2 = t_1$.\vspace{0.5cm}
        \begin{fitch}
          \fj t_1 = t_2 \\
          \fa t_1 = t_1 & \eqi\\
          \fa t_2 = t_1 & \eqe{\varphi\,\text{is}\,x = t_1}{1}{2}
        \end{fitch}
      \end{column}
      \begin{column}{0.48\textwidth}
      Lets prove that if $t_1 = t_2$ and $t_2 = t_3$, then $t_1 = t_3$.\vspace{0.5cm}
        \begin{fitch}
          \fa t_1 = t_2 \\
          \fj t_2 = t_3 & \eqi\\
          \fa t_1 = t_3 & \eqe{\varphi\,\text{is}\,t_1 = x}{2}{2}
        \end{fitch}
      \end{column}
    \end{columns}
\end{slide}


\section*{Exercises on FOL Natural Deduction}


\begin{frame}
\frametitle{Proving that $\exists x \neg\varphi \vdash \neg\forall x\varphi$}

\begin{fitch}
    \fj \exists x \neg\varphi \\
    \fa \fj \forall x\varphi \\
    \fa \fa u\,\,\fj\neg\varphi[u/x] \\
    \fa \fa \,\,\,\,\,\fa \varphi[u/x] & \foralle{2} \\
    \fa \fa \,\,\,\,\,\fa \bot & \falsei{3}{4} \\
    \fa \fa \bot & \exe{1}{3-5}\\
    \fa \neg\forall x\varphi & \negi{2{-}6}
    
\end{fitch}
\end{frame}

\begin{frame}
\frametitle{Proving that $\forall x \varphi \land \psi \vdash \forall x (\varphi \land \psi)$ and $x$ is not free in $\psi$}

\begin{fitch}
    \fj \forall x \varphi \land \psi \\
    \fa \forall x\varphi & \conjel{1} \\
    \fa \psi & \conjer{1} \\
    \fa u\,\,\fj\varphi[u/x] \\
    \fa \:\,\,\,\,\fa \varphi[u/x] \land \psi & \conji{4}{3} \\
    \fa \:\,\,\,\,\fa (\varphi \land \psi)[u/x] & $x \text{ free in } \psi$ \\
    \fa \forall x (\varphi \land \psi) & \foralli{4}{6}
    
\end{fitch}
\end{frame}


\begin{frame}
\frametitle{Proving that $\forall x (\varphi \land \psi) \vdash \forall x \varphi \land \psi$ and $x$ is not free in $\psi$}

\begin{fitch}
    \fj \forall x (\varphi \land \psi) \\
    \fa u\,\,\fa (\varphi \land \psi)[u/x] & \foralle{1}\\
    \fa \:\,\,\,\,\fa \varphi[u/x] \land \psi & $x \text{ not free in } \psi$ \\
    \fa \:\,\,\,\,\fa \psi & \conjer{3}\\
    \fa \:\,\,\,\,\fa \varphi[u/x] & \conjel{3}\\
    \fa \forall x\varphi & \foralli{2}{5} \\
    \fa \forall x\varphi \land \psi & \conji{6}{4}

    
\end{fitch}
\end{frame}




\end{document}
