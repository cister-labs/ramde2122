\documentclass[aspectratio=169]{beamer}

\input{macros/beamerconf}
\usepackage{etex} % fixes new-dimension error
\usepackage{lmodern} % fixes warnings
\usepackage{textcomp}% fixes warnings

\usepackage{graphicx,amsmath}
\usepackage{stmaryrd} % cf. interleave
%\usepackage{./macros/myisolatin1}
%\usepackage{./macros/prooftree}
\usepackage{alltt}
%\usepackage{./macros/circle}
\usepackage{listings}
\usepackage{relsize} % relative size fonts
\usepackage[normalem]{ulem} % strikethrough text (with \sout{.})
\usepackage{tikz}
\usetikzlibrary{%
  positioning
 ,patterns
 ,arrows
 ,automata
 ,calc
 ,shapes
 ,fit
 ,fadings
 ,decorations.pathreplacing
 ,plotmarks
% ,pgfplots.groupplots
 ,decorations.markings
}
% \tikzset{shorten >=1pt,node distance=2cm,on grid,auto,initial text={},inner sep=2pt}
\tikzstyle{aut}=[shorten >=1pt,node distance=2cm,on grid,auto,initial text={},inner sep=2pt]
\tikzstyle{st}=[circle,draw=black,fill=black!10,inner sep=3pt]
\tikzstyle{sst}=[rectangle,draw=none,fill=none,inner sep=3pt]
\tikzstyle{final}=[accepting]
\usepackage[normalem]{ulem} % striking out text with \sout{...}
\usepackage{xspace}

\usepackage{transparent}

% Nicer TT fonts
\usepackage[scaled=.83]{beramono}
\usepackage[T1]{fontenc}




%------ using eurosym -------------------------------------------------------
\usepackage{eurosym}
\def\inh#1{\mbox{\small\euro}_{#1}}
\def\eith#1#2{\mathopen{[}#1 \ ,#2\mathclose{]}}

%------ using xy ------------------------------------------------------------
\usepackage[all]{xy}
%\def\larrow#1#2#3{\xymatrix{ #3 & #1 \ar[l] _-{#2} }}
\def\larrow#1#2#3{\xymatrix{ #3 & #1 \ar[l] _--{#2} }}
\def\rarrow#1#2#3{\xymatrix{ #1 \ar[r]^-{#2} & #3 }}
\def\arLaw#1#2#3#4#5{
\xymatrix{
        #1      \ar@/^1pc/[rr]^-{#4} &
        #5 &
        #2      \ar@/^1pc/[ll]^-{#3}
}}
\def\arLeq#1#2#3#4{\arLaw{#1}{#2}{#3}{#4}\leq}
%------ using pstricks (rnode etc) ------------------------------------------
%\usepackage{pstricks,pst-node,pst-text,pst-3d}
\input{macros/macros_drp}

\usepackage{macros/fitchalt}

%\newcommand{\fitchr}[2]{\ensuremath{#1\text{\sc #2}}}
%
%\newcommand{\conji}[2]{\ensuremath{\land\text{\textbf{\textsc{I}}}(#1,#2)}}
%\newcommand{\conjel}[1]{\ensuremath{\land\text{\textbf{\textsc{E}}}_l(#1)}}
%\newcommand{\conjer}[1]{\ensuremath{\land\text{\textbf{\textsc{E}}}_r(#1)}}
%
%\newcommand{\disjil}[1]{\ensuremath{\lor\text{\textbf{\textsc{I}}}_l(#1)}}
%\newcommand{\disjir}[1]{\ensuremath{\lor\text{\textbf{\textsc{I}}}_r(#1)}}
%\newcommand{\disje}[3]{\ensuremath{\lor\text{\textbf{\textsc{E}}}(#1,#2,#3)}}
%
%\newcommand{\negi}[1]{\ensuremath{\neg\text{\textbf{\textsc{I}}}(#1)}}
%\newcommand{\nege}[1]{\ensuremath{\neg\text{\textbf{\textsc{E}}}(#1)}}
%
%\newcommand{\falsei}[2]{\ensuremath{\bot\text{\textbf{\textsc{I}}}(#1,#2)}}
%\newcommand{\falsee}[1]{\ensuremath{\bot\text{\textbf{\textsc{E}}}(#1)}}
%
%\newcommand{\impi}[2]{\ensuremath{{\to}\text{\textbf{\textsc{I}}}(#1{-}#2)}}
%\newcommand{\impe}[2]{\ensuremath{{\to}\text{\textbf{\textsc{E}}}(#1,#2)}}
%
%\newcommand{\eqi}{\ensuremath{{=}\text{\textbf{\textsc{I}}}}}
%\newcommand{\eqe}[3]{\ensuremath{{=}\text{\textbf{\textsc{E}}}(#1,#2,#3)}}
%
%\newcommand{\foralli}[2]{\ensuremath{{\forall}\text{\textbf{\textsc{I}}}}(#1{-}#2)}
%\newcommand{\foralle}[1]{\ensuremath{{\forall}\text{\textbf{\textsc{E}}}(#1)}}
%
%\newcommand{\exi}[1]{\ensuremath{{\exists}\text{\textbf{\textsc{I}}}}(#1)}
%\newcommand{\exe}[2]{\ensuremath{{\exists}\text{\textbf{\textsc{E}}}(#1{-}#2)}}
%
%\newcommand{\fas}{\;\,\,\,\fa}
%\newcommand{\fal}[1]{{#1}\,\,\fa}


\usepackage{macros/bussproofs}
\usepackage{bm}


\setLecture{2}{RAMDE -- Requirements and Model-driven Engineering}

% \title{
% 	RAMDE -- Requirements and Model-driven Engineering 
% 	}
% \author{David Pereira \and Jos\'{e} Proen\c{c}a}
% \institute{CISTER -- ISEP \\ Porto, Portugal}
% \date{MScCCSE 2020/21}


\begin{document}

\frame[plain]{\titlepage}


\section{Recalling Natural Deduction Rules in First Order Logic}

\begin{frame}[shrink=0.95]{Which are the new rules (on top of Propositional Logic)?}
  \begin{block}{Elimination rule for $\forall$}
  If we know that $\forall x, \varphi$ holds, then we can conclude that $\varphi$ holds for a specific term $t$
  \begin{prooftree}
      \AxiomC{$\forall x\,\varphi$}
      \RightLabel{$\forall\text{\textbf{\textsc{E}}}$}
      \UnaryInfC{$\varphi[t/x]$}
    \end{prooftree}  
  \end{block}
  
  \begin{block}{Introduction rule for $\forall$}
  If we assume some term $t$ and we are able to prove that $\varphi[t/x]$ then we can conclude that $\forall x, \varphi$.
  \begin{prooftree}
      \AxiomC{$[t]$}
        \noLine
        \UnaryInfC{$$}
        \noLine
        \UnaryInfC{$$}
        \noLine
        \UnaryInfC{$$}
        \noLine
        \UnaryInfC{$$}

      \AxiomC{$$}
        \noLine
        \UnaryInfC{$$}
        \noLine
        \UnaryInfC{$\vdots$}
        \noLine
        \UnaryInfC{$$}
        \noLine
        \UnaryInfC{$$}

        
      \AxiomC{$$}
        \noLine
        \UnaryInfC{$$}
        \noLine
        \UnaryInfC{$$}
        \noLine
        \UnaryInfC{$$}
        \noLine
        \UnaryInfC{$\varphi[t/x]$}

      \RightLabel{$\forall\text{\textbf{\textsc{I}}}$}
      
      \TrinaryInfC{$\forall x\,\varphi[t/x]$}
    \end{prooftree}
  \end{block}
  
\end{frame}


\begin{frame}[shrink=0.95]{Which are the new rules (on top of Propositional Logic)?}
  \begin{block}{Elimination rule for $\exists$}
  If we know that $\exists x, \varphi$ holds, and if assuming term $t$ and $\varphi[t/x]$ we can deduce $\psi$, then we can prove $\psi$ overall.
  \begin{prooftree}
      \AxiomC{$\exists x\,\varphi$}
      \AxiomC{$[t\:\:\:\varphi[t/x]]$}
      \noLine
      \UnaryInfC{$\vdots$}
      \noLine
      \UnaryInfC{$\psi$}
      \RightLabel{$\exists\text{\textbf{\textsc{E}}}$}
      \BinaryInfC{$\psi$}
    \end{prooftree}  
  \end{block}
  
  \begin{block}{Introduction rule for $\exists$}
  If we assume some term $t$ and we are able to prove that $\varphi[t/x]$ then we can conclude that $\forall x, \varphi$.
  \begin{prooftree}
      \AxiomC{$\varphi[t/x]$}
      \RightLabel{$\exists\text{\textbf{\textsc{I}}}$}
      \UnaryInfC{$\exists x, \varphi$}  
  \end{prooftree}
  \end{block}
  
\end{frame}





\begin{frame}[shrink=0.95]{Which are the new rules (on top of Propositional Logic)?}
  \begin{block}{Elimination rule for $=$}
  If we know that two terms $t_1$ and $t_2$ are equal and that $\varphi[t_1/x]$ holds, then $\varphi[t_1/x]$ must also hold.
  \begin{prooftree}
      \AxiomC{$t_1 = t_2$}
      \AxiomC{$\varphi[t_1/x]$}
      \RightLabel{${=}\text{\textbf{\textsc{E}}}$}
      \BinaryInfC{$\varphi[t_2/x]$}
    \end{prooftree}  
  \end{block}
  
  \begin{block}{Introduction rule for $=$}
  If we assume some term $t$ and we are able to prove that $\varphi[t/x]$ then we can conclude that $\forall x, \varphi$.
  \begin{prooftree}
      \AxiomC{$$}
      \RightLabel{${=}\text{\textbf{\textsc{I}}}$}
      \UnaryInfC{$t = t$}  
  \end{prooftree}
  \end{block}
  
\end{frame}

\section*{Practical Exercises}

\begin{slide}{Ex1. Build a proof of $\forall x (R(x) \land Q(x)) \vdash \forall x R(x) \land \forall x Q(x)$}

$$
\begin{nd}
  \hypo{1}{\forall x (R(x) \land Q(x))} \pause
  \open[v]
    \have{2}{R(v) \land Q(v)} \Ae{1}\pause
    \have{3}{R(v)} \ae{2}\pause
  \close
  \have{4}{\forall x R(x)} \Ai{2-3}\pause
  \open[v]
    \have{5}{R(v) \land Q(v)} \Ae{1}\pause
    \have{6}{Q(v)} \ae{5}\pause
  \close
  \have{6}{\forall x Q(x)} \Ai{5-6}\pause
  \have{7}{\forall x R(x) \land \forall x Q(x)} \ai{4,7}\pause
\end{nd}
$$

\end{slide}

\begin{slide}{Ex2. Build a proof of $\forall x (R(x) \to Q(x)) \vdash \forall x R(x) \to \forall x Q(x)$}

$$
\begin{nd}
  \hypo{1}{\forall x (R(x) \to Q(x))} \pause
    \open
      \hypo{2}{\forall x R(x)}\pause
        \open[v] 
          \have{3}{R(v) \to Q(v)} \Ae{1}\pause
          \have{4}{R(v)} \Ae{2}\pause
          \have{5}{Q(v)} \ie{3,4}\pause
        \close
        \have{6}{\forall x Q(x)} \Ai{3-5}\pause
    \close
    \have{7}{\forall x P(x) \to \forall x Q(x)} \ii{2-6}\pause
\end{nd}
$$

\end{slide}

\begin{slide}{Ex3. Build a proof of $\exists x (R(x) \to Q(x)) \vdash \exists x R(x) \to \exists x Q(x)$}

$$
\begin{nd}
  \hypo{1}{\exists x (R(x) \to Q(x))} \pause
    \open
      \hypo{2}{\exists x R(x)}\pause
        \open[v] 
          \hypo{3}{R(v) \to Q(v)} \pause
          \open
            \hypo{4}{R(v)} \pause
            \have{5}{Q(v)} \ie{3,4}\pause
            \have{6}{\exists x Q(x)} \Ei{5}\pause
          \close
        \have{7}{\exists x Q(x)} \Ee{2,4-6}\pause
        \close
        \have{8}{\exists x Q(x)} \Ee{1,3-7}\pause
    \close
    \have{9}{\exists x P(x) \to \exists x Q(x)} \ii{2-8}\pause
\end{nd}
$$

\end{slide}

\begin{slide}{Ex4. Build a proof of $\exists x\neg Q(x) \vdash \neg\forall x Q(x)$}

$$
\begin{nd}
  \hypo{1}{\exists x\neg Q(x)}\pause
  \open
    \hypo{2}{\forall x Q(x)}\pause
    \open[v]
        \hypo{3}{\neg Q(v)}\pause
        \have{4}{Q(v)} \Ae{2} \pause
        \have{5}{\bot}  \by{$\bot$I}{3,4}\pause
    \close
    \have{6}{\bot}  \Ee{1,3-5}\pause
  \close
  \have{7}{\neg\forall x Q(x)} \ni{2-6}\pause
\end{nd}
$$
\end{slide}

\begin{frame}[shrink]{Ex5. Build a proof of $\neg \exists x\neg Q(x) \vdash \forall x Q(x)$}

$$
\begin{nd}
  \hypo{1}{\neg \exists x\neg Q(x)}\pause
  \open
    \hypo{2}{\neg\forall x Q(x)}\pause
    \open[v]
      \open
        \hypo{3}{\neg Q(v)}\pause
        \have{4}{\exists x \neg Q(x)} \Ei{3} \pause
        \have{5}{\bot}  \by{$\bot$I}{1,4}\pause
      \close
      \have{6}{\neg\neg Q(v)} \ni{3-5}\pause
      \have{7}{Q(v)} \ne{6}\pause
    \close
    \have{8}{\forall x Q(x)}  \Ai{3-7}\pause
    \have{9}{\bot}  \by{$\bot$I}{2,8}\pause
  \close
  \have{10}{\neg\neg\forall x Q(x)} \ni{2-9}\pause
  \have{11}{\forall x Q(x)} \ne{10}\pause
\end{nd}
$$
\end{frame}

\begin{slide}{Ex6. Build a proof of $\forall x Q(x) \vdash  \neg \exists x\neg Q(x)$}

$$
\begin{nd}
  \hypo{1}{\forall x Q(x)}\pause
  \open
    \hypo{2}{\exists x \neg Q(x)}\pause
    \have{3}{\neg\forall x Q(x)} \by{Ex4.}{1}\pause
    \have{4}{\bot} \by{$\bot$I}{1,3}\pause
  \close
  \have{5}{\neg\exists x Q(x)} \ni{2-4}\pause
\end{nd}
$$
\end{slide}

\begin{frame}[shrink=0.95]{Ex7. Build a proof of $\exists x (P(x) \land Q(x)), \forall x (P(x) \to R(x)) \vdash \exists x (R(x) \land P(x))$}

$$
\begin{nd}
  \have{1}{\exists x (P(x) \land Q(x))}
  \hypo{2}{\forall x (P(x) \to R(x))}\pause

  \open[v]
      \hypo{3}{P(v) \land Q(v)}\pause
      \have{4}{P(v)} \ae{3}\pause
      \have{5}{P(v) \to R(v)} \Ae{2}\pause
      \have{6}{R(v)} \ie{4,5}\pause
      \have{7}{Q(v)} \ae{3}\pause
      \have{8}{R(v) \land Q(v)} \ai{6,7}\pause
      \have{9}{\exists x(R(x) \land Q(x))} \Ei{8}\pause
  \close
  \have{10}{\exists x(R(x) \land Q(x))} \Ee{1,3-9}\pause
\end{nd}
$$

\end{frame}

\begin{frame}[shrink=0.95]{Ex8. Build a proof of $\forall x \forall y ( x = y \to f(x) = f(y))$}

$$
\begin{nd}
  \open[u]
    \open[v]
      \open
      \hypo{1}{u = v}\pause
      \open
        \hypo{2}{\neg(f(u) = f(v))}\pause
        \have{3}{f(u) = f(u)} \by{$=$I}{}\pause
        \have{4}{f(u) = f(v)} \by{$=$E}{1,3}\pause
        \have{5}{\bot} \by{$\bot$I}{2-4}\pause
      \close
      \have{6}{\neg\neg(f(u) = f(v))} \ni{2-5}\pause
      \have{7}{f(u) = f(v)} \ne{6}\pause
      \close
      \have{8}{u = v \to f(u) = f(v)} \ii{1-7}\pause
    \close
    \have{9}{\forall y (u = y \to f(u) = f(y)} \Ai{8}\pause
  \close
  \have{10}{\forall x\forall y (x = y \to f(x) = f(y)} \Ai{9}\pause

\end{nd}
$$

\end{frame}

\begin{frame}[shrink=0.95]{Ex9. Build a proof of $\forall x P(a,x,x), \forall x \forall y \forall z (P(x,y,z) \to P(f(x),y,f(z))) \vdash P(f(a),a,f(a))$}

$$
\begin{nd}
  \have{1}{\forall x P(a,x,x)}\pause
  \hypo{2}{\forall x \forall y \forall z (P(x,y,z) \to P(f(x),y,f(z)))}\pause
  \have{3}{P(a,a,a)} \Ae{1}\pause
  \have{4}{\forall y \forall z (P(a,y,z) \to P(f(a),y,f(z)))} \Ae{2}\pause
  \have{5}{\forall z (P(a,a,z) \to P(f(a),a,f(z)))} \Ae{4}\pause
  \have{6}{P(a,a,a) \to P(f(a),a,f(a))} \Ae{5}\pause
  \have{7}{P(f(a),a,f(a))} \ie{3,6}\pause
  

\end{nd}
$$

\end{frame}


\begin{frame}[shrink=0.95]{Ex10. Build a proof of $ \exists x \exists y (H(x,y) \lor H(y,x)), \neg \exists x H(x,x) \vdash \exists x\exists y \neg(x=y)$}

$$
\begin{nd}
  \have{1}{\exists x \exists y (H(x,y) \lor H(y,x))}
  \hypo{2}{\neg \exists x H(x,x)}\pause
  \open[u]
    \open[v]
      \open
        \hypo{3}{u = v}\pause
        \open 
          \hypo{4}{H(u,v) \lor H(v,u)}\pause
          \open
            \hypo{5}{H(u,v)}
            \have{6}{H(u,u)} \by{$=$E}{3,5}\pause
            \have{7}{\exists x H(x,x)} \Ei{5}\pause
            \have{8}{\bot} \by{$\bot$I}{2,7}\pause
          \close
        \close
      \close
    \close
  \close
\end{nd}
$$

\end{frame}

\begin{frame}[shrink=0.95]{Ex10. Build a proof of $ \exists x \exists y (H(x,y) \lor H(y,x)), \neg \exists x H(x,x) \vdash \exists x\exists y \neg(x=y)$}

$$
\begin{nd}
  \open
    \open
      \open
        \open 
          %\have[\vdots]{skip1}{\vdots}
          \have[][8]{8}{\vdots}\pause
          \open
            \hypo{9}{H(v,u)}\pause
            \have{10}{H(v,v)} \by{$=$E}{3,9}\pause
            \have{11}{\exists x H(x,x)} \Ei{10}\pause
            \have{12}{\bot} \by{$\bot$I}{2,7}\pause
          \close
        \have{13}{\bot} \oe{4,5-8,9-12}\pause
        \close
        \have{14}{\bot} \Ee{1,4-13}\pause
      \close
      \have{15}{\neg(u = v)} \ni{3-14}\pause
    \close
    \have{16}{\exists y \neg(u = y)} \Ei{15}\pause
  \close
  \have{17}{\exists x \exists y \neg(x = y)} \Ei{16}\pause
\end{nd}
$$
\end{frame}

\begin{frame}[shrink=0.95]{Ex11. Build a proof of $ \exists y \exists x Q(y,x) \vdash \exists x\exists y Q(y,x)$}

$$
\begin{nd}
  \hypo{1}{\exists y \exists x Q(y,x)}\pause
  \open[u]
    \hypo{2}{\exists x Q(u,x)}\pause
    \open[v]
      \hypo{3}{Q(u,v)}\pause
      \have{4}{\exists y Q(y,v)} \Ei{3}\pause
    \close
    \have{5}{\exists y Q(y,v)} \Ee{2,3-4}\pause
    \have{6}{\exists x\exists y Q(y,x)} \Ei{5}\pause
  \close
  \have{7}{\exists x\exists y Q(y,x)} \Ee{1,2-6}\pause
\end{nd}
$$
\end{frame}

\begin{frame}[shrink=0.95]{Ex12. Build a proof of $\vdash (\exists x P(x) \to \forall x R(x)) \to (\forall x (P(x) \to Q(x)))$}

$$
\begin{nd}
  \open
    \hypo{1}{\exists x P(x) \to \forall x R(x)} \pause
    \open[u]
      \open
        \hypo{2}{P(u)}
        \have{3}{\exists x P(x)} \Ei{2} \pause
        \have{4}{\forall x R(x)} \ie{1,3} \pause
        \have{5}{R(u)} \Ae{4} \pause
      \close
      \have{6}{P(u) \to Q(u)} \ii{2-5} \pause
    \close
    \have{7}{\forall x (P(x) \to Q(x))} \Ai{2-6} \pause
  \close
  \have{8}{(\exists x P(x) \to \forall x R(x)) \to (\forall x (P(x) \to Q(x)))} \ii{1-7} \pause
\end{nd}
$$
\end{frame}

\begin{frame}[shrink=0.95]{Ex13. Build a proof of $\vdash x = f(y) \to \forall z (P(x,z) \to P(f(y),z))$}

$$
\begin{nd}
  \open
    \hypo{1}{x = f(y)} \pause
    \open[u]
      \open
        \hypo{2}{P(x,u)} \pause
        \have{3}{P(f(y),u)} \by{$=$E}{1,2} \pause
      \close
      \have{4}{P(x,u) \to P(f(y),u)} \ii{2-3} \pause
    \close
    \have{5}{\forall z (P(x,z) \to P(f(y),z))} \Ai{2-4} \pause
  \close
  \have{8}{x = f(y) \to \forall z (P(x,z) \to P(f(y),z))} \ii{1-5} \pause
\end{nd}
$$

\end{frame}

\begin{frame}[shrink=0.95]{Ex14. Build a proof of $\forall x\forall y\forall u\forall v(x = u \to (y = v \to f(x,y) = f(u,v)))$}

$$
\begin{nd}
  \open[a]
    \open[b]
      \open[c]
        \open[d]
          \open
          \hypo{1}{a=b} \pause
          \open
            \hypo{2}{c=d} \pause
            \have{3}{f(a,c) = f(a,c)} \by{$=$I}{} \pause
            \have{4}{f(a,c) = f(b,c)} \by{$=$E}{1,3} \pause
            \have{5}{f(a,c) = f(b,d)} \by{$=$E}{2,4} \pause
          \close
          \have{6}{c=d \to f(a,c) = f(b,d)} \ii{2-5} \pause
        \close
        \have{7}{a = b \to (c=d \to f(a,c) = f(b,d))} \ii{1-6} \pause
        \close
        \have{8}{\forall v(a = b \to (c=v \to f(a,c) = f(b,v)))} \Ai{7} \pause
      \close
      \have{9}{\forall u\forall v(a = u \to (c=v \to f(a,c) = f(u,v)))} \Ai{8} \pause
    \close
    \have{10}{\forall y\forall u\forall v(a = u \to (y=v \to f(a,y) = f(u,v)))} \Ai{9} \pause
  \close
  \have{11}{\forall x\forall y\forall u\forall v(x = u \to (y=v \to f(x,y) = f(u,v)))} \Ai{10} \pause
\end{nd}
$$

\end{frame}

\begin{frame}[shrink=0.95]{Ex15. Build a proof of $\exists x\forall y(P(x) \land (P(y) \to y=x)) \vdash \forall x\forall y((P(x) \land P(y)) \to x = y)$}
$$
\begin{nd}
  \hypo{1}{\exists x\forall y(P(x) \land (P(y) \to y=x))} \pause
  \open[a]
    \open[b]
      \open
        \hypo{2}{P(a) \land P(b)} \pause
          \open
            \hypo{3}{P(a) \land (P(b) \to b=a)} \pause
            \have{4}{P(b)} \ae{2} \pause
            \have{5}{P(b) \to b = a} \ae{3} \pause
            \have{6}{b = a} \ie{4,5} \pause
            \have{7}{b = b} \by{$=$I}{} \pause
            \have{8}{a = b} \by{$=$E}{6,7} \pause
          \close
          \have{9}{a = b} \Ee{1,3-8} \pause
      \close
      \have{10}{(P(a) \land P(b)) \to a = b} \ii{2-9} \pause
    \close  
    \have{11}{\forall y ((P(a) \land P(y)) \to a = y)} \Ai{10} \pause
  \close
  \have{12}{\forall x\forall y ((P(x) \land P(y)) \to x = y)} \Ai{11} \pause
\end{nd}
$$

\end{frame}

\end{document}
