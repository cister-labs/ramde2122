\documentclass[11pt]{article}


\input{macros}
\date{RAMDE -- 2021/2022}

\usepackage[latin1]{inputenc} % pt characters

\usepackage{draftwatermark}
\SetWatermarkText{Draft}
\SetWatermarkScale{1.4}

\begin{document}
 
% --------------------------------------------------------------
%                         Start here
% --------------------------------------------------------------
 
\title{A3: Analysing behaviour\\
{\Large Crossing the river -- Part 2}}

\author{David Pereira \& Jos\'{e} Proen\c{c}a} 


\maketitle

\myparagraph{To do:} Produce a report as a PDF document including the answers to the exercises below.

\myparagraph{To submit:} The PDF report and a new \bash{vending.mcrl2} file (for Exercise~\ref{ex:vm}), placed in your group's git repository. ALL students should push commits.

\myparagraph{Deadline:} TBD % 14 Dec 2020 @ 23:59 (Monday)

\myparagraph{Auxiliary files:} You will need the 3 files produced in your last assignment: \bash{farmer1.mcrl2}, \bash{farmer2.mcrl2} and \bash{farmer3.mcrl}.


% --------------------------------------------------------------
\section*{Verification of the farmer-fox-goose-beans problem}

Recall the specifications in the \bash{farmer1}, \bash{farmer2}, and \bash{farmer3} projects from the modelling exercises (\url{https://cister-labs.github.io/ramde2122/assignments/a2-modelling.pdf})
You will now verify properties of these systems.
In \bash{mcrl2ide}, a property can be written using \bash{Tools>Add Property}.
There are 2 types of properties: \textbf{Equivalence} and \textbf{Mu-Calculus}, covered by this assignment.

% --------------------------------------------------------------
\section*{LTS Equivalence}


\begin{myExercise}  
Create variations of the \code{Sys} processes in \bash{farmer1} and \bash{farmer2} and compare them to the originals as follows.

% Recall the action-based \bash{farmer1} specification from Exercise~\ref{ex:ba1} and the state-based \bash{farmer2} specification from Exercise~\ref{ex:ba2}.


\subex{Create a new process \code{SysHide} in both \bash{farmer1} and \bash{farmer2} equal to \code{Sys} but {hiding all allowed actions except \code{win}} (using \code{hide}).
\textbf{Show the resulting \code{SysHide} processes for each file.}
}

\subex{
Combine both specifications of \bash{farmer1} and \bash{farmer2} in a single specification.
Rename \code{Sys} from \bash{farmer1} to \code{Sys1} and \code{SysHide1}, and similarly for \code{Sys} from \bash{farmer2}.
Redefine the function \code[morekeywords={ok}]{ok} by setting it to true, i.e., define \code[morekeywords={ok}]{ok(fm,f,g,b)=true;}.

Visualise the processes \code{SysHide1} and \code{SysHide2}. Compare them using strong bisimulation by adding a new \textbf{Equivalence} property that compares them. %Note that you will have to delete the \code{init} block.
\textbf{What can you conclude?}}


% \subex{
% Visualise the minimised LTS for \code{SysHide2} with respect to branching bisimulation, using the tool in \bash{Tools>Show reduced State Space}.
% \textbf{Include a screenshot of the minimised LTS and explain what information can we infer from this LTS.}}


\end{myExercise}



\section*{Verification of properties}


\begin{myExercise}
Answer the questions below on the use of mu-calculus for specifying properties in mCRL2.

\subex{\label{ex:ver1}
What does the property ``\code{[true*]<ready>true}'' mean? Does it hold in any of these 2 LTSs?}

\subex{\label{ex:ver2}
Does the property ``\code{[true*.foxr.win]false}'' holds for \bash{farmer1}? Does the equivalent property ``\code{[true*.fox(right).win]false}'' holds for \bash{farmer2}? What can you conclude?}

% \subex{\label{ex:ver3}
% Recall that \bash{farmer1} is less complete than \bash{farmer2}, because it fails to include some important invariants.
% Write a \textbf{single property} for \bash{farmer2} to capture that:
% \begin{itemize}
%   \item no bad state is reached, and
%   \item the goal is reached (everyone can cross).
% \end{itemize}
% \textbf{Verify} it using mCRL2 toolset.
% \textbf{Verify} if its equivalent formulation holds for \bash{farmer1}.
% }

\subex{
Consider now the extended system \bash{farmer3}.
In this example there is a an extra process called \code{Counter(n:Nat)}.
\textbf{Define the following two properties} over actions of this counter:
\begin{enumerate}
\item It is possible to win after exactly 7 moves.
\item It is not possible to win in less than 7 moves.
\end{enumerate}
}
\end{myExercise}
 

\section*{Modelling a vending machine}

\begin{myExercise} \label{ex:vm}
Specify two interacting processes in mCRL2:
\begin{itemize}
  \item \textbf{a vending machine} with 2 products, apples and bananas, costing 1eur and 2eur respectively; and
  \item \textbf{a user} who can insert 1eur or 2eur coins and request for products.
\end{itemize}
Provide two variations of this system and include them in files \bash{vending1.mcrl2} and \bash{vending2.mcrl2}, respectively, according to the requirements below. Try to keep the specifications simple. \textbf{Submit this file in your git repository.}
%You have the freedom to select the set of actions and behaviour that you believe to make 

\subex{Specify in \bash{vending1.mcrl2} a system such that the properties below hold.}
\vspace{-6mm}
\begin{lstlisting}
[true*.pay2eur.pay2eur] false
[true*.pay2eur.pay1eur] false
[true*.pay2eur]<(!pay1eur && !pay2eur)*.getApple
<true*.pay2eur.true*.getBanana> true \end{lstlisting}
\textbf{Show your specification} and \textbf{show a screenshot of its LTS}.

\vspace{2mm}
\subex{Specify another system in \bash{vending2.mcrl2} such that the properties below hold.}
\vspace{-6mm}
\begin{lstlisting}
<true*.pay2eur.pay2eur> true
<true*.getApple> true
<true*>[true*.getApple] false\end{lstlisting}
\textbf{Show your specification} and \textbf{show a screenshot of its LTS}.

\end{myExercise}


% \section*{SysML project}
% \begin{myExercise} 
%   Recall the mCRL2 model that you specified in the last exercise of Part 1 of this assignment.
%   \subex{\label{ex:s1}
%     Specify 4 relevant and different properties in mCRL2 of the model that specified in the previous part.
%     Include an informal description for each property, and obey the following restrictions:
%     \begin{itemize}
%       \item all properties should hold in your system
%       \item at least one starts with \code{<...>}
%       \item at least one starts with \code{[...]}
%       \item at least one has a trace that gives evidence of the validity of the formula.
%     \end{itemize}
%   }

%   \subex{\label{ex:s2}
%     Specify 2 \textbf{useful} formulas that your system \textbf{does not} support, including an informal description for each formula.
%     At least one formula should have a counter-example that proves the formula.
%   }

%   \subex{Show the evidence of \ref{ex:s1} and the counter-example of \ref{ex:s2}.}
% \end{myExercise}


%%%%%%%%%%%%%%%%%%%%%%%%%

\section*{Self-peer-evaluation}
\begin{myExercise}
  In a scale from 0-5, where 5 is better than 0, give a mark to you and each of your team groups for each of the following criteria:
  \begin{itemize}
    \item \textbf{Effort} (time spent)
    \item \textbf{Quality} (of the work produced)
    \item \textbf{Collaboration} (how easy it was to meet and interact)
  \end{itemize}
  \textbf{Send this information individually} by e-mail or via Teams to David Pereira and Jos\'{e} Proen\c{c}a.
\end{myExercise}


% --------------------------------------------------------------
%     You don't have to mess with anything below this line.
% --------------------------------------------------------------
 
\end{document}