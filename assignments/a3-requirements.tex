\documentclass[11pt]{article}


\input{macros}
\date{RAMDE -- 2021/2022}

\usepackage[utf8x]{inputenc} % pt characters and eur sign (instead of latin1)

% \usepackage{draftwatermark}
% \SetWatermarkText{Draft}
% \SetWatermarkScale{1.4}

\begin{document}
 
% --------------------------------------------------------------
%                         Start here
% --------------------------------------------------------------
 
\title{A3: Requirement analysis}
% {\Large Crossing the river -- Part 2}}

\author{David Pereira \& Jos\'{e} Proen\c{c}a \& Eduardo Tovar} 


\maketitle

\descrbox{To do}{
 Produce a report as a PDF document including the answers to the exercises below.
}

\descrbox{What to submit}{
 Submit in your group's git repository:
 (1) the PDF report, (2) the Doorstop requirements requested in the exercises, and (3) the HTML documentation produced by Doorstop.
}

\descrbox{Deadline:}{
  9 Jan 2022 @ 23:59 (Monday) -- together with Assignment 2
}




\section*{Requirements for a vending machine}


\begin{myExercise} \label{ex:vm}
Recall the vending machine from the last assignment.
The informal assignments over the machine were given by the text below.

\begin{quote}\it
  I would like the vending machine to sell 3 items: apples, bananas, and chocolates.
  It should be possible to buy chocolates for 2€ and fruit for 1€.
  Only 1€ and 2€ coins are accepted.
  The machine has a maximum capacity for 1€ coins and for 2€ coins. 
  The machine does not accept coins if its capacity is full.
  The machine should give change back when buying fruit after inserting 2€.
  If the machine has already 2€ inserted, it refuses another coin.
  If the machine has no 1€ coins, it cannot not sell fruit with a 2€ coin. 
  The user can request the money back after inserting coins.
\end{quote}


\subex{\label{ex:vm1}
   \textbf{Create and classify} the requirements from the text above following the EARS patterns}

%\subex{\label{ex:vm2}
%   \textbf{Classify the requirements} from the previous exercise according to type of requirements of the EARS approach.
%}

\subex{\label{ex:vm3}
  \textbf{Use the Doorstop tool} to produce these requirements in the git repository, under the folder \bash{req/vending}.
  Generate the \textbf{html documentation} based on these requirements.
  \\[2mm]
  Include both the Doorstop requirements and the html documentation in your group's git repository.
}
\end{myExercise}


\section*{Requirements for the Farmer's problem}

\begin{myExercise} \label{ex:farmer}

 Recall the farmer's problem from the 2 previous assignments.

\subex{\label{ex:f1}
  \textbf{Create and classify} a set of requirements using the EARS patterns that capture there constraints and goals.
}

\subex{\label{ex:f2}
  \textbf{Use the Doorstop tool} to produce these requirements in the git repository, under the folder \bash{req/farmer}.
  Generate the \textbf{html documentation} based on these requirements (in the same way you did in  Exercise~\ref{ex:vm3}.
  \\[2mm]
  Include both the Doorstop requirements and the html documentation in your group's git repository.
}


\end{myExercise}


% \section*{SysML project}
% \begin{myExercise} 
%   Recall the mCRL2 model that you specified in the last exercise of Part 1 of this assignment.
%   \subex{\label{ex:s1}
%     Specify 4 relevant and different properties in mCRL2 of the model that specified in the previous part.
%     Include an informal description for each property, and obey the following restrictions:
%     \begin{itemize}
%       \item all properties should hold in your system
%       \item at least one starts with \code{<...>}
%       \item at least one starts with \code{[...]}
%       \item at least one has a trace that gives evidence of the validity of the formula.
%     \end{itemize}
%   }

%   \subex{\label{ex:s2}
%     Specify 2 \textbf{useful} formulas that your system \textbf{does not} support, including an informal description for each formula.
%     At least one formula should have a counter-example that proves the formula.
%   }

%   \subex{Show the evidence of \ref{ex:s1} and the counter-example of \ref{ex:s2}.}
% \end{myExercise}


%%%%%%%%%%%%%%%%%%%%%%%%%

\section*{Self-peer-evaluation}
\begin{myExercise}
  In a scale from 0-5, where 5 is better than 0, give a mark to you and each of your team groups for each of the following criteria:
  \begin{itemize}
    \item \textbf{Effort} (time spent)
    \item \textbf{Quality} (of the work produced)
    \item \textbf{Collaboration} (how easy it was to meet and interact)
  \end{itemize}
  \textbf{Send this information individually} as before by email or Teams to David Pereira and Jos\'{e} Proen\c{c}a. No justification is needed -- e.g., \emph{``Group 3: Jo\~{a}o: Effort 5, Quality 4, Collaboration 5; Maria: ...''}.
\end{myExercise}


% --------------------------------------------------------------
%     You don't have to mess with anything below this line.
% --------------------------------------------------------------
 
\end{document}