\documentclass[11pt]{article}


\input{macros}
\date{RAMDE -- 2021/2022}

\usepackage[latin1]{inputenc} % pt characters
\usepackage{enumitem}
\usepackage{draftwatermark}
\SetWatermarkText{Draft}
\SetWatermarkScale{1.4}


\newcommand{\myset}[1]{\{#1\}}

\begin{document}
 
% --------------------------------------------------------------
%                         Start here
% --------------------------------------------------------------
 
\title{A1: Set Theory}%\\
%{\Large Crossing the river -- Part 1}}

\author{David Pereira \& Jos\'{e} Proen\c{c}a} 


\maketitle

\vspace*{-5mm}
\myparagraph{To do:} Produce a a PDF document including the answers to the selected exercises from below. In each question, there are exercises marked with a $\star$ from which you can select. In each question, the precise number of selectable exercises for you to submit, is defined.

\myparagraph{To submit:} The PDF report containing your solutions for the exercises you have selected for each of the questions. 

\myparagraph{Deadline:} TBD %6 Dec 2021 @ 23:59 (Monday)

%\myparagraph{Auxiliary files:} \url{https://cister-labs.github.io/ramde2122/assignment/farmer.zip}


\section*{Recalling the main definitions of set theory}
Before proceeding with the exercises, lets recall the main definitions of set theory that are necessary to solve them. Essentially, you need to know the following:
\begin{itemize}
  
  \item a set is a collection of elements, and can be finite or infinite. In the case the set is finite, we call its size the cardinality of the set, and we denote it by $|S|$ (or, alternatively, by $\#S$)
  
  \item if $x$ is an element of a given set $S$, then we say that "$x$ is in $S$" and we denote this by $x \in S$
  
  \item a set can be defined by comprehension, that is, we consider a property $\mathcal{P}$ that determines if a value is part, or not of the set. We denote this by $\myset{x\,|\,\mathcal{P}}$
  
  \item if $A$ and $B$ are sets, and all elements of $A$ are also elements of $B$, then we say that "$A$ is a subset of $B$, and denote this relation by $A \subseteq B$
  
  \item if $A$ and $B$ are sets, then $A \cup B$ is the union of the elements of each set, i.e., the set formed by all elements that belong to $A$ and all the elements that belong to $B$
  
  \item if $A$ and $B$ are sets, then $A \cap B$ is the intersection of the elements of each set (i.e., the values that belong to both sets)
  
  \item if $A$ and $B$ are sets, then $A \setminus B$ is the set formed by the elements of $A$ that are not elements of $B$. We call this "set difference".

  \item if $A$ is a set, we call the complement of $A$ the set formed by all elements that are not part of $A$. We denote this by $\overline{A}$
\end{itemize}

\section*{Set theory: membership, subsets, union, intersection, and difference}

%\vspace*{-3mm}
%\lstinputlisting{farmer1.mcrl2}

\begin{myExercise} \label{ex:ba1}

  Which of the statements are true and which are false?
  \begin{enumerate}[label=\alph*)]
    \item $3 \in \myset{1,2,3}$
    \item $\{3\} \in \{\{1\},\{2\},\{3\}\}$
    \item $3 \subseteq \{3\}$
    \item $\emptyset \subseteq \emptyset$
    \item $\emptyset \in \emptyset$
    \item $\emptyset \in \myset{\emptyset}$
  \end{enumerate}

\end{myExercise}

~\\[-6mm]

%----------------------------------------------
\begin{myExercise} \label{ex:ba2}

  Consider the following sets: $A = \myset{2n\,|\,n \in \mathbb{N}}$, $B = \myset{3n\,|\,n \in \mathbb{N}}$, and $C = \myset{6n\,|\,n \in \mathbb{N}}$.
  
  \subex{Which of the following statements are false?
  \begin{enumerate}[label=\alph*)]
    \item $B \subseteq C$
    \item $C \subseteq B$
    \item $\myset{3,8} \subseteq A \cup B$
    \item $\myset{3,8} \subseteq A \cap C$
    \item $A \cap B = \emptyset$
    \item $???$
  \end{enumerate}}
  
  \subex{Calculate the following sets:
    \begin{enumerate}[label=\alph*)]
    \item $A \cap B$
    \item $B \cup C$
    \item $A \setminus B$
    \item $B \setminus C$  
    \end{enumerate}
  }
  
\end{myExercise}

~\\[-6mm]

\begin{myExercise}
Consider the following sets: $A = \myset{1,2,3,4,5}$, $B = \myset{1,2,4,8}$, $C = \myset{1,2,3,5,7}$, and $D = \myset{2,4,6,8}$. Calculate the following sets:
\begin{enumerate}[label=(\alph*)]
\item $(A \cup B)  \cap C$
\item $A \cup (B \cap C)$
\item $(A \cap B) \cup (C \cap D)$
\item $(A \cup B) \cap (C \cup D)$
\item $(A \cup B) \setminus C$
\item $A \cup (B \setminus C)$
\item $(A \setminus B) \setminus C$
\item $A \setminus (B \setminus C)$
\item $(B \setminus C) \setminus D$
\item $D \setminus (B \setminus C)$
\end{enumerate}

\end{myExercise}

~\\[-6mm]

\begin{myExercise}
  
  Find the sets $A$ and $B$ that satisfy the conditions imposed (each question is independent):
  
  \begin{enumerate}[label=(\alph*)]
    \item $A \setminus B = \myset{1,3,7,11}$, $B \cup A = \myset{1,2,3,5,6,7,11,14}$, and $A \cap B = \myset{2}$  
    \item $A \setminus B = \myset{1,3,7,11}$, $B \setminus A = \myset{2,6,8}$, and $A \cap B = \myset{4,9}$  
    \item $A \setminus B = \myset{1,2,4}$, $B \setminus A = \myset{7,8}$, and $A \cup B = \myset{1,2,4,5,7,8,9}$  
  \end{enumerate}

\end{myExercise}

~\\[-6mm]

\begin{myExercise}
  For each of the statements presented below, answer if they are true and, when they are false provide a counter-example:
  \begin{enumerate}[label=(\alph*)]
  \item if $A \subseteq B$ and $B \subseteq C$, then $A \subseteq C$
  \item if $A \subseteq B$ and $B \not\subseteq C$, then $A \not\subseteq C$
  \item if $A \subseteq B$ and $B \not\subseteq C$, then $A \subseteq C$
  \item if $A \cap C = B \cap C$, then $A = B$
  \item if $A \cup C = B \cup C$, then $A = B$
  \item if $A \subseteq B$ if, and only if, $A \cap \overline{B} = \emptyset$
  \end{enumerate}

\end{myExercise}

~\\[-6mm]

\begin{myExercise}
  Provide an explanation for the following statements that are true, or provide a counter-example for those that are false.
  
  \begin{enumerate}[label=(\alph*)]
    \item $A \cap B = B \cap A$  
    \item $(A \setminus B) \cap C = (A \cap C) \setminus B$
    \item $A \setminus (B \cap C) = (A \setminus B) \cap (A \setminus C)$
    \item $A \cap (B \cup C) = (A \cap B) \cup (A \cap C)$
    \item if $A \subseteq B$ then $A \cap B = A$

  \end{enumerate}

\end{myExercise}


%\section*{Self-peer-evaluation}
%\begin{myExercise}
%  In a scale from 0-5, where 5 is better than 0, give a mark to you and each of your team groups for each of the following criteria:
%  \begin{itemize}
%    \item \textbf{Effort} (time spent)
%    \item \textbf{Quality} (of the work produced)
%    \item \textbf{Collaboration} (how easy it was to meet and interact)
%  \end{itemize}
%\end{myExercise}


% % --------------------------------------------------------------
% \section*{LTS Equivalence}


% \begin{myExercise}  
% Recall the action-based \bash{farmer1.mcrl2} specification from Exercise~\ref{ex:ba1} and the state-based \bash{farmer2.mcrl2} specification from Exercise~\ref{ex:ba2}.


% \subex{Modify the initial process (\code{init}) of both \bash{farmer1.mcrl2} and \bash{farmer2.mcrl2} to {hide all allowed actions except \code{win}} (using \code{hide}), and save them as \bash{farmer1-tau.mcrl2} and \bash{farmer2-tau.mcrl2}, respectively.
% In \bash{farmer2-taus.mcrl2} {redefine the function \code[morekeywords={ok}]{ok}} by setting it to true, i.e., define \code[morekeywords={ok}]{ok(fm,f,g,b)=true;}.
% \textbf{Show the resulting \code{init} block from each file.}
% }

% \subex{
% Generate the \bash{.lts} files corresponding to \bash{farmer1-tau.mcrl2} and \bash{farmer2-tau.mcrl2}, and compare them using strong bisimulation using the following command.
% \textbf{What can you conclude?}}

% \begin{lstlisting}[style=bash]
% $\texttt{\$}$ ltscompare --equivalence=bisim farmer1-taus.lts farmer2-taus.lts
% \end{lstlisting}


% \subex{
% Using \bash{ltsconvert}, minimise the LTS for \bash{farmer2-taus.mcrl2} with respect to branching bisimulation, using the command below.
% \textbf{Include a screenshot of the minimised LTS and explain what information can we infer from this LTS.}}

% \begin{lstlisting}[style=bash]
% $\texttt{\$}$ ltsconvert --equivalence=branching-bisim farmer2-taus.lts
% \end{lstlisting}


% \end{myExercise}



 
\end{document}